\chapter{Rings and Fields}

\section{Rings and Fields}

\begin{definition}[Ring]
    A \emph{ring} $\langle R, +, \cdot\rangle$ is a set $R$ together with two binary operations $+$ and $\cdot$ which we call \emph{addition} and \emph{multiplication} defined on $R$ such that the following are satisfied: \begin{enumerate}[label = ($\mathscr{R}_{\arabic*}$)]
        \item $\langle R, +\rangle$ is an abelian group.
        \item Multiplication is associative.
        \item For all $a, b, c \in R$, the \emph{left and right distributive laws} – $a\cdot(b+c)=(a\cdot b)+(a\cdot c)$ and $(a+b)\cdot c = (a\cdot c)+(b\cdot c)$ hold.
    \end{enumerate}
\end{definition}
\begin{example}
    $\Z, \Q, \R, \C$ are all rings with addition and multiplication. In fact, these axioms hold for any subset of the complex numbers that is a group under addition and closed under multiplication.
\end{example}
\begin{example}
    For any ring $R$, the collection of all $n\times n $ matrices having elements of $R$ as entries, $M_n(R)$, is an abelian additive group. Note, in particular, that (matrix) multiplicaton is not commutative for these.
\end{example}
\begin{theorem}
    If $R$ is a ring with additive identity 0, then for any $a,b \in R$, we have: \begin{enumerate}
        \item $0a = a0 =0.$
        \item $a(-b) = (-a)b = -(ab)$.
        \item $(-a)(-b)=ab.$
    \end{enumerate}
\end{theorem}
\begin{proof}
    (i) $a0+a0 = a(0+0) = a0 = 0+a0$ so $a0 = 0.$ (ii) $a(-b) + ab = a(0) = 0$ so $a(-b) = -(ab)$. The same goes for $(-a)b$. (iii) $-(a(-b))=-(-(ab))$ so $(-a)(-b)=ab.$
\end{proof}
\begin{definition}[Ring Homomorphism]
    For rings $R$ and $R'$, a map $\phi\colon R\to R'$ is a homomorphism if both $\phi(a+b) = \phi(a)+\phi(b)$ and $\phi(ab) = \phi(a)\phi(b)$. $\phi$ is one-to-one if and only if its kernel ($\{a \in R\mid \phi(a) = 0'\}$) is just the subset $\{0\}$ of $R$. This gives rise to a factor group as well as a factor ring.
\end{definition}
\begin{definition}[Ring Isomorphism]
    A ring isomorphism is a homomorphism $\phi\colon R\to R'$ that is bijective. Group isomorphisms do not necessarily extend to ring isomorphisms.
\end{definition}
\begin{definition}[Unity]
  A ring with a multiplicative identity element, denoted by 1, is a \emph{ring with unity}. 1 is the "unity."
\end{definition}
\begin{definition}[Commutative Ring]
    A ring in which multiplication is commutative is a \emph{commutative ring}.
\end{definition}
\begin{example}
    For intergers $r, s$ where $\gcd(r,s) = 1$, the rings $\Z_{rs}$ and $\Z_r\times\Z_s$ are isomorphic. $\phi\colon Z_{rs}\to \Z_r \times \Z_s$ defined by $\phi(n\cdot1)=n\cdot(1,1)$ is an additive group isomorphism. Also, $\phi(nm)=(nm)\cdot(1,1)=[n\cdot(1,1)][m\cdot(1,1)]=\phi(n)\phi(m)$ so it is a ring isomorphism as well.
\end{example}
\begin{definition}[Multiplicative Inverse]
    A \emph{multiplicative inverse} of an element $a$ in a ring $R$ with unity $1 \neq 0$ is an element $a^{-1} \in R$ so $aa^{-1}=a^{-1}a=1$.
\end{definition}
\begin{remark}
    Only the ring $\{0\}$ has both the multiplicative and additive inverse as the same element.
\end{remark}
\begin{definition}[Unit, Division Rings]
    Let $R$ be a ring with $1 \neq 0$. An element $u \in $ is a \emph{unit} of $R$ if it has a multiplicative inverse in $R$. If every nonzero element is a unit, then $R$ is a \emph{division ring} or \emph{skew field}. 
\end{definition}
\begin{definition}[Field]
    A \emph{field} is a commutative division ring. A noncommutative division ring is a \emph{strictly skew field}.
\end{definition}
\begin{definition}[Subring and Subfield]
    A \emph{subring} is a subset of a ring with under induced operations. A subfield is defined similarly. 
\end{definition}
\begin{note}
    \emph{Unit} denotes an element with a multiplicative inverse and \emph{unity} denotes the actual multiplicative identity element 1.
\end{note}

\section{Integral Domains}

\begin{definition}[Divisors of 0]
    If $a$ and $b$ are two nonzero elements of a ring $R$ so that $ab = 0$, then $a$ and $b$ are \emph{divisors of 0}.
\end{definition}
\begin{theorem}
    In the ring $\Z_n$, the divisors of 0 are the nonzero elements that are \emph{not} relatively prime to $n$.
\end{theorem}
\begin{proof}
    Let $m \in Z_n, m \neq 0$ and $d = gcd(m,n) \neq 1$. Thus, $m\left(\frac{n}{d}\right) = \left(\frac{m}{d}\right)n$ so $(\frac{m}{d})n$ is 0 in $\Z_n$ so $m(n/d)$ is 0 in $\Z_n$ also but neither $m, n/d = 0$ so $m$ is a divisor of 0.

    On the other hand, if $m \in \Z_n, \gcd(m,n) = 1$ and $ms = 0$ for some $s \in Z_n$, then $n \mid ms$. But, $\gcd(m,n)=1$ so $ n\mid s$ but $s \in Z_n$ so $s = 0$ in $\Z_n$ meaning $m$ is not a divisor.
\end{proof}
\begin{corollary}
    If $p$ is prime, then $\Z_p$ has no divisors of 0.
\end{corollary}
\begin{theorem}
    The multiplicative cancellation laws hold in a ring $R$ if and only if $R$ has no divisors of 0.
\end{theorem}
\begin{proof}
    Say $R$ is a ring with cancellation laws and $ab = 0$ for some $a,b \in R$. If $a \neq 0$, then $ab = a0$ implies $b = 0$ via cancellation, WLOG. Conversely, if $R$ has no divisors of 0 and $ab = ac, a \neq 0$ for any $a,b,c \in R$, then $0 = ab - ac = a(b-c)$. $a=0$ and $R$ has no divisors of 0 so $b = c$ and we can do cancellation. The same goes for right cancellation.
\end{proof}
\begin{definition}[Integral Domain]
    A \emph{integral domain} $D$ is a commutative ring with unity $1 \neq 0$ that has \emph{NO} divisors of 0.
\end{definition}
\begin{theorem}
    Every field $F$ is an integral domain.
\end{theorem}
\begin{proof}
    For any $a, b \in F$, if $a \neq 0$ and $ab = 0$ then $b = 1b = (a^{-1}a)b = a^{-1}0 = 0.$ So no divisors of 0 in $F$ exist (from commutativtity for other direction).
\end{proof}
\begin{theorem}
    Every finite integral domain is a field.
\end{theorem}
\begin{proof}
    Take the finite domain $D$ with finite elements $0, 1, a_1, \ldots, a_n$. We must show that for any $a \in D, a \neq 0, \exists b \in D$ so $ab =1$. If all elements of $D$ are distinct and all nonzero (no divisors of 0), then we find $a1, aa_1, \ldots aa_n$ can contain no 0 elements but must all be distinct as if they weren't, by cancellation laws, $aa_i = aa_j \implies a_i = a_j$. Thus, this must be some permutation of $0, 1, a_1, \ldots, a_n$ so some $a_k$ must be the multiplicative inverse of 1.
\end{proof}
\begin{corollary}
    If $p$ is prime, then $\Z_p$ is a field.
\end{corollary}
\begin{definition}[Characteristic of a Ring]
    The \emph{characteristic of a ring} $R$ is the least positive integer $\min\{n \in \Z^+ \mid n\cdot a = 0 \text{ for all } a \in R\}.$ If none exists, the characteris of $R$ is 0.
\end{definition}
\begin{example}
    The ring $\Z_n$ has characteristic $n$ while $\Z, \Q, \R, \C$ all have characteristic 0.
\end{example}
\begin{theorem}
    Let $R$ be a unital ring. If $n \cdot 1 \neq 0$ for all $n \in \Z^+$, then $R$ has characteristic 0. But, if $n \cdot 1 = 0$ then the smallest such integer $n$ is the characteristic of $R$.
\end{theorem}
\begin{proof}
    If $n \cdot 1 \neq 0$ for all $n \in \Z^+$, then surely we cannot have $n \cdot a = 0$ for all positive integers $n$ so $R$ has characteristic 0. Otherwise, if $n \cdot 1 = 0$ for some $n \in \Z^+$, then for any $a \in R$, $n \cdot a = a + \cdots + a = a(1 + \cdots + 1) = a(n \cdot 1) = a0 = 0.$
\end{proof}

\section{Fermat's and Euler's Theorems}

\begin{remark}
    For any field, the nonzero elements form a group under the field multiplication.
\end{remark}
\begin{theorem}{Fermat's Little Theorem}
    If $a \in \Z$ and $p$ is a prime \emph{not} dividing $a$, then $p$ divides $a^{p-1}$ so $a^{p-1} \equiv 1 \text{(mod $p$)}$ for $a \neq 0 (\text{mod} p)$.
\end{theorem}
\begin{corollary}
    If $a \in \Z$, then for any prime $p$, $a^p \equiv a \text{(mod $p$)}.$
\end{corollary}
\begin{example}
    $8^{103} \div 13$ gives $(8^{12})^8(8^7) \equiv (1^8)(8^7) \equiv (-5)^7 \equiv (25)^3(-5) \equiv (-1)^3(-5) \equiv 5 \text{(mod 13)}.$
\end{example}
\begin{theorem}
    The set $G_n$ of nonzero elements of $\Z_n$ that are not 0 divisors forms a group under multiplication modulon $n$.
\end{theorem}
\begin{proof}
    For any $a, b \in G_n$, if $ab \notin G_n$ then there would exist some $c \neq 0$ in $\Z_n$ so $(ab)c = 0$. But, this implies $a(bc) = 0$. Because $b$ is not a 0 divisor, $bc \neq 0$ but then $a \notin G_n$. Contradiction, so $ab \in G_n$ so $G_n$ has closure. $1 \in G_n$ obviously and multiplication mod $n$ is associative.

    To show the existence of an inverse, we can use a proof by counting. For any $a \in G_n$, given distinct elements of $G_n$: $1, a_1, \ldots, a_r$, the elements $a1, aa_1, \ldots, aa_r$ must also be distinct as $aa_i = aa_j \implies a(a_i - a_j) = 0$ but $a$ is not a divisor of 0 so $a_i = a_j$. Because of closure, these products must cover $G_n$ so there exists some $a_k$ so $aa_k = 1$.
\end{proof}
\begin{remark}[Euler's Totient/Phi Function $\phi(n)$]
    $\phi(n)$ is equal to the number of positive intergers less than or equal to $n$ and relatively prime to $n$. Note $\phi(1) = 1$. This is equal to the number of nonzero elements of $\Z_n$ that are not divisors of 0.
\end{remark}
\begin{theorem}[Euler's Theorem]
    If $a$ is an integer relatively prime to $n$, then $a^{\phi(n)}-1$ is divisible by $n$. I.e. $a^{\phi(n)} \equiv 1 (\text{mod } n)$.
\end{theorem}
\begin{proof}
    If $a$ is coprime with $n$ then the coset $a + n\Z$ of $n\Z$ contaning $a$ contains an integer $b <n$ also coprime to $n$. Because multiplication mod $n$ of representatives is well-defined, $a^{\phi(n)} \equiv b^{\phi(n)} (\text{mod } n)$. $b$ can then be viewed as an element of $G_n$ of order $\phi(n)$ consisting of the $\phi(n)$ elements of $\Z_n$ coprime to $n$ so $b^{\phi(n)} \equiv 1 (\text{mod}  n).$
\end{proof}
\begin{theorem}
    Let $m \in \Z^+, a \in \Z_m$ so $\gcd(a,m) = 1$. For each $b \in \Z_m$, the equation $ax = b$ has a unique solution in $\Z_m$.
\end{theorem}
\begin{proof}
    $a$ is a unit in $\Z_m$ by the previous theorem so $s= a^{-1}b$ is a solution of this equation. multiplying both sides of $ax=b$ by $a^{-1}$ reveals this indeed is the only solution.
\end{proof}
\begin{theorem}
    Let $m \in \Z^+$ and $a, b \in \Z_m$. Let $d = \gcd(a,m).$ The equation $ax = b$ has a solution in $\Z_m$ iff $d \mid b$. If so, the equation has exactly $d$ solutions in $\Z_m$.
\end{theorem}
\begin{proof}
    Suppose $s \in Z_m$ is a solution to $ax = b$. Then $as - b = qm$ for some $q \in \Z$ so $b = as - qm$. $d$ divides $a, m$ so $d$ must also divide the LHS so a solution $s$ only exists if $d \mid b$.

    Next, if $d \mid b$, let $a = a_1d, b=b_1d, m=m_1d$ so $as - b = qm$ can be rewritten as $d(a_1s - b_1) = dqm_1$ so $as-b$ is a multiple of $m$ if and only if $a_1s - b_1$ is also a multiple of $m_1$. This yelds the solutions $s \in \Z_m$ of $ax = b$ as precisely $s, s+m_1, s+2m_1, \ldots, s+(d-1)m_1.$ Thus, $d$ solutions to the equation exist in $\Z_m$.
\end{proof}
\begin{example}
    Take the congruence $12x \equiv 27 (\text{mod } 18)$. The greatest common divisor of 12 and 18 is 6, but 6 is not a divisor of 27 so no such solutions exist.

    For $15x \equiv 27 (\text{mod } 18)$, however, their gcd is 3 which divides 27 so this has 3 solutions, $3+ 18\Z,9+ 18\Z,15+ 18\Z.$
\end{example}

\section{The Field of Quotients of an Integral Domain}

\begin{remark}
    Let's think of the rationals as the formal quotient $(a,b)$ within $D \times D$ for integral domain $D = \Z.$
\end{remark}
\begin{definition}[Equivalent]
    Let $S = \{(a,b) \mid a,b \in D, b \neq 0\}$ for an integral domain $D$. Two elements $(a,b)$ and $(c,d)$ in $S$ are \emph{equivalent}, denoted as $(a,b) \sim (c,d)$ if and only if $ad=bc.$
\end{definition}
\begin{lemma}
    The relation $\sim$ on the set $S$ is an equivalence relation. (i) $ab = ba \implies (a,b) \sim (b,a).$ (ii) $(a,b) \sim (c,d) \implies ad = bc \implies cb = da \implies (c,d) \sim (b,a).$ (iii) $(a,b) \sim (c,d), (c,d) \sim (e,f) \implies ad = bc, cf = ed$ so $acf/e = bc$ so $af = be$ implying $(a,b) \sim (e,f).$ (Note division is simply shorthand for cancellation which is allowed because of integral domain).
\end{lemma}
\begin{note}
    This chapter discusses the formation of field $F$ from $D \times D$. Proof shows addition and multiplication well defined and has field axioms and contains $D$.
\end{note}
\begin{lemma}
    To show that $F$ contains $D$, we simply construct an isomorphism $i\colon D\to F$ as given by $i(a) = [(a,1)]$ with a subring of $F$.
\end{lemma}
\begin{proof}
    For any $a,b \in D$, $i(a+b) = [(a+b, 1)] = [(a1+b1,1)] = [(a,1)] + [(b,1)] = i(a) + i(b).$ Also, $i(ab) = [(ab,1)] = [(a,1)][(b,1)] = i(a)i(b).$ Thus, $i$ is a ring homomorphism. Next, if $i(a) = i(b)$, then $[(a,1)] = [(b,1)] \implies (a,1) \sim (b,1) \implies a1 = 1b \implies a=b$ so $i$ is injective. Because it is of the same size as $D$, this is an isomorphism of $D$ with $i[D].$ So $i[D]$ is a subdomain of $F$.
\end{proof}
\begin{theorem}[Field of Quotients of $D$]
    Any integral domain $D$ can be enlarged to or embedded in a field $F$ so each element of $F$ can be expressed as a quotient of two elements of $D$. Here, a field $F$ is a \emph{field of quotients of} $D$.
\end{theorem}
\begin{proof}
    $[(a,b)] = [(a,1)][(1,b)] = [(a,1)]/[(b,1)] = i(a)/i(b).$
\end{proof}
\begin{theorem}
    Let $F$ be a field of quotients of $D$ and $L$ be any field containing $D$. Then, there exists a map $\psi\colon F\to L$ which gives an isomorphism of $F$ with a subfield of $L$ so $\psi(a) = a$ for all $a \in D$.
\end{theorem}
\begin{proof}
    Proof omitted.
\end{proof}
\begin{corollary}
    Every field $L$ containing an integral domain $D$ contains a field of quotients of $D$.
\end{corollary}
\begin{corollary}
    Any two fields of quotients of an integral domain $D$ are isomorphic.
\end{corollary}

\section{Rings of Polynomials}

\begin{note}
    We call $x$ an \emph{indeterminate} rather than a variable in the ring $\Z[x].$
\end{note}
\begin{definition}[Polynomial $f(x)$ with Coefficients in $R$]
    Let $R$ be a ring. A \emph{polynomial} $f(x)$ with coefficients in $R$ is an infinite formal sum $\Sigma_{i=0}^\infty a_ix^i$ where $a_u \in R$ and $a_i=0$ for all but a finite number of values of $i$. The largest such value of $i$ is the \emph{degree} of $f(x)$ while $a_i$ are the coefficients.
\end{definition}
\begin{note}
    An element of $R$ is a \emph{constant polynomial}.
\end{note}
\begin{theorem}
    The set $R[x]$ of all polynomials in an indeterminate $x$ with coefficients in a ring $R$ is a ring under polynomial addition and multiplication. If $R$ is commutative, then so is $R[x]$ and $R$ has unity $1 \neq 0$ so $1$ is also a unity for $R[x].$
\end{theorem}
\begin{proof}
    Clearly $\langle R[x], + \rangle$ is an abelian group. The associative law for multiplication and the distributive laws are clear as well.
\end{proof}
\begin{example}
    In $\Z_2[x],$ $(x+1)^2 = (x+1)(x+1) = x^2 + (1+1)x + 1 = x^2 + 1$ while $(x+1) + (x+1) = 0x.$
\end{example}
\begin{example}
    We can even form the ring $(R[x])[y]$, i.e. the ring of polynomials in $y$ with coefficents that are polynomials in $x$. This is naturally isomorphic to $(R[y])[x]$. Thus, we can denote the ring $R[x,y]$ as the ring of of polynomials in two indeterminates $x$ and $y$ with coefficients in $R$. In fact the ring $R[x_1, \ldots, x_n]$ of polynomials in $n$ indeterminates $x_i$ with coefficients in $R$ is similarly defined. 
    
    Given integral domain $D$, $D[x]$ is also an integral domain. If $F$ is a field, $F[x]$ is a field but \emph{not} a field as $x$ is not a unit in $F[x]$. However, we can do same goes for $F(x_1, \ldots, x_n)$ or the field of rational functions with $n$ indeterminates over field $F$.
\end{example}
\begin{theorem}[The Evaluation Homomorphisms for Field Theory]
    Let $F$ be a subfield of a field $E$, $\alpha \in E,$ and $x$ be the indeterminate. Define the map $\phi_\alpha\colon F\to E$ as $\phi_\alpha(a_0+a_1x + \cdots + a_nx^n) = a_0 + a_1\alpha + \cdots + a_n\alpha^n$ is a homomorphism of $F[x]$ into $E$. Note that $\phi_\alpha(x) = \alpha$ so $\phi_\alpha$ maps $F$ isomorphically by the identity map such that $\phi_\alpha(a) = a$ for any $a \in F$. This map is the \emph{evaluation homomorphism at $\alpha$}.
\end{theorem}
\begin{proof}
    This map is obviously well-defined. Next, for any $f(x) = a_0 + a_1x + \cdots + a_nx^n, g(x) = b_0 + b_1x + \cdots + b_mx^m,$ let $h(x) = f(x) + g(x) = c_0 + c_1x + \cdots + c_rx^r.$ Thus, $\phi_\alpha(f(x) + g(x)) = \phi_\alpha(h(x)) = c_0 + c_1\alpha + \cdots + c_r\alpha^r = a_0 + a_1x + \cdots + a_nx^n + b_0 + b_1x + \cdots + b_mx^m = \phi_\alpha(f(x)) + \phi_\alpha(g(x)).$ Multiplication works similarly by definition of polynomial multiplication $d_j = \Sigma_{i = 0}^j a_ib_{j-i}.$ Thus, $\phi_\alpha$ is a homomorphism.
\end{proof}
\begin{example}
    Let $F = \Q, E = \R$ and apply the evaluation homomorphism $\phi_0\colon Q[x] \to \R$ such that each polynomial is mapped onto its constant term.
\end{example}
\begin{example}
    Let $F = \Q, E = \C$, we can apply the evaluation homomorphism from $Q[x] \to \C$ at $i$ so $\phi(x^2 + 1) = 0$ so $x^2+1$ is in the kernel of $\phi_i.$
\end{example}
\begin{remark}
    A more interesting example uses the same evaluation homomorphism from $\Q[x] \to \R$ but at $\pi$. Because $\pi$ is transcendental, no algebraic solution exists for $a_0 + a^1\pi + \cdots + a_n\pi^n = 0$ as this implies $a_i = 0$ so the kernel of $\phi_\pi$ is $\{0\}$ implying it is an injective map and thus ring isomorphic to $\Q[x].$
\end{remark}
\begin{definition}[Zero of $f(x)$]
    Take subfield $F$ of field $E$, $\alpha \in E$ and let $f(x) = a_0 + a_1x + \cdots + a_nx^n \in F[x]$. Given evaluation homomorphism $\phi_\alpha\colon F[x] \to E$, we say $\alpha$ is a \emph{zero} of $f(x)$ if $f(\alpha) = 0.$
\end{definition}
\begin{theorem}
    The polynomial $x^2 - 2$ has no zeroes in the rational numbers. Thus $\sqrt{2} \notin \Q.$
\end{theorem}
\begin{proof}
    Take $m/n$ for $m, n \in \Z$ such that $(m/n)^2 = 2$ and we simplify so that $\gcd(m,n) = 1.$ Then, $m^2 = 2n^2$ but this implies 2 is a factor of $2n^2$ and therefore must be a factor of $m^2$ as well. But, if this is the case, $m^2$ is a multiple of 4 so $n^2$ must have a multiple of 2 as well. But this implies their greatest common divisor is not 1. Contradiction.
\end{proof}

\section{Factorization of Polynomials over a Field}

\begin{theorem}[Division Algorithm for $F\text{[}x\text{]}$]
    Let $f(x) = a_nx^n + a_{n-1}x^{n-1} + \cdots + a_0$ and $g(x) = b_mx^m + b_{m-1}x^{m-1} + \cdots + b_0$ be two elements of $F[x]$ with nonzero $a_n, b_m \in F, m > 0.$ Then there exist unique polynomials $q(x), r(x)$ in $F[x]$ so $f(x) = g(x)q(x) + r(x)$ where either $r(x) = 0$ or its degree is less than the degree $m$ of $g(x)$.
\end{theorem}
\begin{theorem}[Factor Theorem]
    An element $a \in F$ is a zero of $f(x) \in F[x]$ if and only if $x-a$ is a factor of $f(x)$ in $F[x].$
\end{theorem}
\begin{proof}
    $\implies\colon$Suppose that $f(a) = 0$ for some $a \in F$. Then, there exists a $q(x), r(x) \in F[x]$ so $f(x) = (x-a)q(x) + r(x)$ where $r(x) = 0$ or the degree of $r(x) < 1.$ hus, $r(x)$ must equal $c$ for $c \in F$ such that $f(x) = (x-a)q(x)+c$. Applying the evaluation homomorphism $\phi_a\colon F[x]\to F$, we get $0 = f(a) = 0q(a)+c$ implying $c=0$. Therefore, $x-a \mid f(x).$
    
    $\impliedby\colon$ If $x-a$ is a factor of $f(x) \in F[x]$, then clearly, $f(x) = (x-a)q(x)$ for $q(x) \in F[x]$ so $f(a) = (0)q(a) = 0$
\end{proof}
\begin{corollary}
    A nonzero polynomial $f(x) \in F[x]$ of degree $n$ can have at most $n$ zeros in a field $F$.
\end{corollary}
\begin{corollary}
    If $G$ is a finite subgroup of the multiplicative group $(F^*,\cdot)$ for a field $F$, then $G$ is cyclic.
\end{corollary}
\begin{proof}
    If $G$ is a finite abelian group, it must be isomorphic to $\Z_{d_1}\times\cdots\times\Z_{d_r}$ where each $d_i$ is a power of a prime. Thinking of each $\Z_{d_i}$ as a multiplicative cyclic group, take $m = \text{lcm}(d_1,d_2\ldots,d_r)$ so $m \leq d_1d_2\cdots d_r$. Note that, for any $\alpha \in G, \alpha^m = 1$ so every element of $G$ is a zero of $x^m-1$. Because $G$ has $d_1d_2\cdots d_r$ elements yet $x^m-1$ has at most $m$ zeros, $m \geq d_1d_2\cdots d_r$ so $m = d_1d_2\cdots d_r$. Therefore, the primes involved in the prime powers are distinct implying the group $G$ is isomorphic to the cyclic group $Z_m$.
\end{proof}
\begin{definition}[Irreducible Polynomial in $F[x\text{]}$]
    A nonconstant polynomial $f(x) \in F[x]$ is \emph{irreducible over $F$} if $f(x) = g(x)h(x)$ for $g, h \in F[x]$ both of lower degree than $f(x)$. Otherwise $f(x)$ is \emph{reducible over $F$}.
\end{definition}
\begin{example}
    Note that $x^2-2$ has no zeros in $\Q$ and is therefore not irreducible over $\Q$ but clearly has roots in $\R$ over which it is reducible.
\end{example}
\begin{theorem}
    Let $f(x) \in F[x]$ and let $f(x)$ have degree 2 or 3. Then, it is reducible over $F$ if and only if it has a zero in $F$.
\end{theorem}
\begin{proof}
    If $f(x)$ is reducible and therefore $f(x) = g(x)h(x)$, we can say $g(x)$, WLOG, has degree 1. Thus, $g(x)$ is of the form $x-a$ so $g(a) = 0$ so $f(a) = 0$ implying $f(x)$ indeed must have3 a zero in $F$. Conversely, if $f(a) = 0$ for some $a \in F$, then $x-a \mid f(x)$ making $f(x)$ reducible.
\end{proof}
\begin{theorem}
    If $f(x) \in \Z[x]$, then $f(x)$ factors into a product of two polynomial of lower degrees $r,s \in \Q[x]$ if and only if it has such a factorization with polynomials of the same degree $r,s \in \Z[x].$
\end{theorem}
\begin{corollary}
    If $f(x) = x^n + a_{n-1}x^{n-1}+\cdots+a_0 \in \Z[x]$ with $a_0 \neq 0,$ and if $f(x)$ has a zero in $\Q$, then it has a zero $m$ in $\Z$ and $m$ must divide $a_0.$
\end{corollary}
\begin{theorem}[Einstein Criterion]
    Let $p \in \Z$ be a prime. Suppose $f(x) = a_nx^n + \cdots + a_0 \in \Z[x]$ and $a_n \neq 0 (\text{mod } p)$, but $a_i = 0 (\text{mod } p)$ for all $i < n$, with $a_0 \neq 0 (\text{mod } p^2)$. Then $f(x)$ is irreducible over $\Q$.
\end{theorem}
\begin{proof}
    We need only show $f(x)$ does not factor into polynomials of lower degree in $\Z[x]$. If $f(x) = (b_rx^r + \cdots + b_0)(c_sx^s + \cdots + c_0)$ is such a factorization with $b_r, c_s \neq 0$ and $r,s < n$, then $a_0 \neq 0 (\text{mod } p^2)$ implies $b_0, c_0$ are not both congruent to 0 mod $p$. Supposing $b_0 \neq 0 (\text{mod } p)$ but $c_0 = 0 (\text{mod } p)$. This then implies, because $a_n \neq 0 (\text{mod } p)$, that $b_r, c_s \neq 0 (\text{mod } p)$. Because $a_n = b_rc_s$, if $m$ is the smallest value of $K$ so $c_k \neq 0 (\text{mod } p)$, then $a_m = b_0 + b_1c_{m-1} + \cdots + \begin{cases} b_mc_0 & \text{if } r \geq m \\ b_rc_{m-r} & \text{if } r < m\end{cases}.$ The fact neither $b_0$ nor $c_m$ are congruent to 0 modulo $p$ while $c_{m-1},\cdots, c_0$ are all congruent to 0 modulo $p$ implies that $a_m \neq 0 (\text{mod } p)$ so $m = n$. Hence, $s = n$ so $s$ is not less than $n$ against our asssumption meaning this factorization was nontrivial. 
\end{proof}
\begin{corollary}[$p^{\text{th}}$ Cyclotomic Polynomial]
    The polynomial $\Phi_p(x) = \frac{x^p-1}{x-1} = x^{p-1} + x^{p-2} + \cdots + x + 1$ is irreducible over $\Q$ for any prime $p$. $\Phi_p(x)$ is the \emph{$p^{\text{th}}$ cyclotomic polynomial}.
\end{corollary}
\begin{theorem}
    Let $p(x)$ be an irreducible polynomial in $F[x]$ If $p(x)$ divides $r(x)s(x)$ for $r,s \in F[x]$, then either $p(x)$ divides $r(x)$ or $s(x)$
\end{theorem}
\begin{theorem}
    If $F$ is a field, then every nonconstant polynomial $f(x) \in F[x]$ can be factored in $F[x]$ into a product of irreducible polynomials which are unique except for order and for unit (nonzero constant) factors in $F$.
\end{theorem}

% \section{Noncommutative Examples}
% \section{Ordered Rings and Fields}