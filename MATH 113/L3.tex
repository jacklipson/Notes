\chapter{Homormorphisms and Factor Groups}

\section{Homomorphisms}

\begin{definition}[Homomorphism]
    A map $\phi$ of a group $G$ into a group $G'$ is a \emph{homomorphism} if the homomorphism property that $\phi(ab)=\phi(a)\phi(b)$ for all $a,b \in G$ holds.
\end{definition}
\begin{remark}[Trivial Homomorphism]
    There is at least always the homomorphism $\phi\colon G\to G'$ defined as $\phi(g) = e'$ for all $g \in G$ is called the \emph{trivial homomorphism}.
\end{remark}
\begin{example}
    Let $S_n$ be the symmetric group on $n$ letters and let $\phi\colon S_n\to\Z_n$ be defined by: \vspace{-10pt}\[\phi(\sigma)=\begin{cases*}0\quad\sigma\text{ even permutation} \\ 1\quad\sigma \text{ odd permutation.}\end{cases*}\] Clearly, $\sigma$ is a homormorphism.
\end{example}
\begin{example}[Evaluation Homomorphism]
    Let $F$ be the additive group of all functions mapping $R$ into $R$ and $R$ be the additive group of all reals and $c \in \R$. Then, $\phi_c\colon F\to\R$ is the \emph{evaluation homomorphism} defined as $\phi_c(f) = f(c)$ for $f \in F$.
\end{example}
\begin{example}
    The \emph{projection map} $\pi_i\colon G\to G_i$ where $G = G_1\times G_2\times\cdots\times G_i\times\cdots\times G_n$ and $\pi_i(g_1,g_2,\cdots,g_i,\cdots,g_n)=g_i$ for each $i = 1,2,\cdots,n$. 
\end{example}
\begin{definition}[Image, Range, Preimage]
    Let $\phi$ be a mapping on a set $X$ into a set $Y$ and $A \subseteq X, B \subseteq Y$. The \emph{image $\phi[A]$ of $A$ in $Y$ under $\phi$} is $\{\phi(a)\mid a \in A\}$.

    The set $\phi[X]$ is the \emph{range of $\phi$}.

    The \emph{inverse image $\phi^{-1}[B]$ of $B$ in $X$} is $\{x\in X \mid \phi(x) \in B\}$.
\end{definition}
\begin{theorem}
    Let $\phi$ be a homomorphism of a group $G$ into a group $G'$. Then,
    \begin{enumerate}
        \item If $e$ is the identity element in $G$, $\phi(e)$ is the identity element $e' \in G'$.
        \item If $a \in G$, then $\phi(a^{-1}) = \phi(a)^{-1}$.
        \item If $H$ is a subgroup of $G$, then $\phi[H]$ is a subgroup of $G'$.
        \item If $K'$ is a subgroup of $G'$, then $\phi^{-1}[K']$ is a subgroup of $G$.
    \end{enumerate}
\end{theorem}
\begin{definition}[Kernel]
    Let $\phi\colon G \to G'$ be a homomorphism of groups. The subgroup $\phi^{-1}[\{e'\}]=\{x\in G\mid \phi(x) = e'\}$ is the \emph{kernel of $\phi$}, denoted by $\Ker(\phi)$.
\end{definition}
\begin{theorem}
    Let $\phi\colon G\to G'$ be a group homomorphism and $H = \Ker(\phi)$. For $a \in G$, the set $$\phi^{-1}[\{\phi(a)\}]=\{x\in G\mid\phi(x)=\phi(a)\}$$ is the left coset $aH$ and right coset $aH$ of $H$. Thus, the partitions of $G$ into left cosets and right cosets are the same.
\end{theorem}
\begin{proof}
    We want to show $\{x \in G \mid \phi(x) = \phi(a)\} = aH$, i.e. they are subsets of one another.
    
    $\subseteq\colon$If $\phi(x) = \phi(a)$, then $e' = \phi(a)^{-1}\phi(x) = \phi(a^{-1})\phi(x) = \phi(a^{-1}x)$ so $a^{-1}x \in H = \Ker(\phi).$ Thus, $a^{-1}x = h$ for some $h \in H$ so $x = ah \in aH$ so $\{x \in G \mid \phi(x) = \phi(a) = aH\}.$

    $\supseteq\colon$ Say $y \in aH$ so $y = ah$ for some $h \in H$. Thus, $\phi(y) = \phi(ay) = \phi(a)\phi(h) = \phi(a)e'=\phi(a)$ so $y \in \{x \in G \mid \phi(x) = \phi(a)\}.$
\end{proof}
\begin{corollary}
    A group homomorphism $\phi\colon G\to G'$ is injective $\iff \Ker(\phi) = \{e\}.$
\end{corollary}
\begin{proof}
    $\implies\colon$If $\Ker(\phi) = \{e\}$, then the elements mapped to $\phi(a)$ are exactly the elements of the left coset $a\{e\} = \{e\}$ showing that $\phi$ is injective.
    $\impliedby\colon$If $\phi$ is injective, then simply $e$ can be the only element mapped to $e'$. 
\end{proof}
\begin{note}[Show $\phi\colon G\to G'$ Is an Isomorphism]
    \leavevmode
    \begin{enumerate}[label=(Step \arabic*), leftmargin=*]
        \item Show $\phi$ homormorphism.
        \item Show $\Ker(\phi) = \{e\}.$
        \item Show $\phi$ is surjective.
    \end{enumerate}
\end{note}
\begin{definition}[Normal Subgroup]
    A subgroup $H$ of a group $H$ is normal if its left and right cosets coincide, that is, if $gH=Hg$ for all $g \in G$. Normal subgroups are denoted as $H \vartriangleleft G$.
\end{definition}
\begin{note}
    All subgroups of abelian groups are normal.
\end{note}
\begin{corollary}
    If $\phi\colon G\to G'$ is a group homomorphism, then $\Ker(\phi)$ is a normal subgroup of $G$. 
\end{corollary}

\section{Factor Groups}

\begin{theorem}
    Let $\phi\colon G\to G'$ be a group homomorphism with kernel $H$. Then the cosets of $H$ form a \emph{factor group} $G/H$ where $(aH)(bH) = (ab)H$. Also, the map $\mu\colon G/H\to\phi[G]$ defined by $\mu(aH) = \phi(a)$ is an isomorphism.

    A factor group $G/H$ is also called the \emph{factor group of $G$ modulo $H$} and elements in the same coset are said to be \emph{congruent modulo $H$}.
\end{theorem}
\begin{example}
    The isomorphism $\mu\colon\Z/5\Z\to\Z_5$ assigns to each coset of $5\Z$ its smallest nonnegative element, i.e. $\mu(5\Z)=0, \mu(1+5\Z)=1$, etc.
\end{example}
\begin{theorem}
    Let $H$ be a subgroup of a group $G$. Then left coset multiplication is well defined by $(aH)(bH)=(ab)H$ if and only if $H$ is a normal subgroup of $G$.
\end{theorem}
\begin{proof}
    $\implies\colon$Suppose $(aH)(bH)=(ab)H$ is a well-defined operation on left cosets. Then, we want to show $aH$ and $Ha$ are the same set. Let $x\in aH$. Picking representatives $x \in aH$ and $a^{-1} \in a^{-1}H$, we get $(xH)(a^{-1}H) = (xa^{-1})H$. This must be equal to $(aH)(a^{-1}H) = (eH) = H$ so $xa^{-1} = h \in H$ Thus, $x = ha \implies x \in Ha$ so $aH\subseteq Ha$. The symmetric proof is also true so $aH = Ha$.

    $\impliedby\colon$ Suppose $H$ is a normal subgroup of $G$. Take $a,ah_1 \in aH, b,bh_2 \in bH$ so $h_1b \in Hb = bH$ so $h_1b=bh_3$ for some $h_1,h_2,h_3 \in H$. Thus, $$(ah_1)(bh_2)=a(h_1b)(h_2)=a(bh_3)h_2=(ab)(h_3h_2) \in (ab)H.$$ Going the other direction, if $x \in (ab)H \implies x = abh = (ae)(bh) \in (aH)(bH).$
\end{proof}
\begin{definition}[Factor/Quotient Group]
    Let $H \vartriangleleft G$. Then the cosets of $H$ form a group $G/H$ under the binary operation $(aH)(bH)=(ab)H$. This group is called the \emph{factor, or quotient, group of $G$ by $H$}.
\end{definition}
\begin{example}
    Because $\Z$ is an abelian group, $n\Z$ is a normal subgroup so $\Z/n\Z$ is isomorphic to $\Z_n$. 
\end{example}
\begin{theorem}
    Let $H \vartriangleleft G$. Then $\gamma\colon G\to G/H$ given by $\gamma(x) = xH$ is a homomorphism with kernel $H$.
\end{theorem}
\begin{proof}
    Let $x,y \in G$. Clearly, $\gamma(a)\gamma(b)=(aH)(bH)=(ab)H=\gamma(ab)$ so it is a homomorphism. Plus, if $\gamma(x) \in eH$, then $xH = eH$ so clearly $x \in H$. Thus, $\Ker(\gamma) = H.$ 
\end{proof}
\begin{theorem}[The Fundamental Homomorphism Theorem]
    Let $\phi\colon G\to G'$ be a group homomorphism with kernel $H$. Then $\phi[G]$ is a group and $\mu\colon G/H\to \phi[G]$ given by $\mu(gH) = \phi(g)$ is an isomorphism. If $\gamma\colon G\to G/H$ is the homormorphism given by $\gamma(g) = gH$, then $\phi(g) = \mu\gamma(g)$ for each $g \in G$. $\mu$ is the \emph{natural, or canonical isomorphism}.
\end{theorem}
\begin{theorem}
    The following are 3 equivalent conditions for a subgroup $H$ of a group $G$ to be a normal subgroup of $G$:
    \begin{enumerate}
        \item $ghg^{-1} \in H$ for all $g \in G, h \in G$.
        \item $gHg^{-1} = H$ for all $g \in G$
        \item $gH = Hg$ for all $g \in G$.
    \end{enumerate}
\end{theorem}
\begin{definition}[(Inner) Automorphism]
    An isomorphism $\phi\colon G\to G$ of a group $G$ with itself is a \emph{automorphism of $G$}. The automorphism $i_g\colon G\to G$ where $i_g(x) = gxg^{-1}$ for all $x \in G$ is the \emph{inner automorphism of $G$ by $g$}.
\end{definition}
\begin{definition}[Conjugate Subgroup]
    Performing $i_g$ on $x$ is called the \emph{conjugation of $x$ by $g$}. A subgroup $K$ of $G$ is a \emph{conjugate subgroup} of $H$ if $K = i_g[H]$ for some $g \in G$.
\end{definition}

\section{Simple Groups}

\begin{remark}
    For a normal subgroup $N$ of $G$, the factor group $G/N$ collapses $N$ to a single element, namely the identity. 
\end{remark}
\begin{example}
    The trivial subgroup $N = \{0\}$ of $\Z$ is obviously normal and has factor group isomorphic to $\Z$.
\end{example}
\begin{example}
    We can show the falsity of the converse of Lagrange's Theorem. That is, $A_4$ has order 12 yet has no subgroup of order 6. 
    
    Suppose $H < A_4$ and $H$ was of order 6. It would follow that $H$ is a normal subgroup of $A_4$ so $A_4/H$ would only have 2 elements, $H$ and $\sigma H$ for some $\sigma \in A_4/H.$ Because it's a group of order 2, the square of this element but be the identity so $(\sigma H)(\sigma H) = H$. Thus, the square of every element in $A_4$ must be in $H$. However, this is 8 elements so $H$ cannot have order 6.
\end{example}
\begin{theorem}
    Let $G = H \times K$ be the direct product of groups $H$ and $K$. Then $\bar{H} = \{(h,e)\mid h \in H\} \vartriangleleft G$. Also, $G/\bar{H} \simeq K$ and $G / \bar{K} \simeq H$ in natural ways.
\end{theorem}
\begin{proof}
    Take the homomorphism $\pi_2\colon H\times K \to K$ where $\pi_2(h,k) = k.$ Because $\Ker(\pi_2) = \bar{H}$, $\bar{H} \vartriangleleft H \times K.$ Because $\pi_2$ is onto $K$, $(H\times K)/\bar{H}\simeq K.$
\end{proof}
\begin{theorem}
    A factor group of a cyclic group is cyclic.
\end{theorem}
\begin{proof}
    Let $G$ be a cyclic group generated by $a$ with normal subgroup $N$. To compute all powers of $aN$ means computing all powers of the representative $a$ which gives all elements in $G$ such that $aN$ gives all cosets of $N$ such that $G/N$ is cyclic.
\end{proof}
\begin{example}
    To find the factor group of $\Z_4\times\Z_6/\langle(2,3)\rangle$, note that $\langle(2,3)\rangle$ has order 2 and $\Z_4\times\Z_3$ has order 24 implying the factor group has order 12 which is either of form, up to isomorphism, $\Z_4\times\Z_3$ or $\Z_2\times\Z_2\times\Z_3$. However, note that $(1,0) + \langle(2,3)\rangle$ is of order 4 in the factor group $(\Z_4\times\Z_6)/\langle(2,3)\rangle$ so the group must be isomorphic to $\Z_4\times\Z_3$ or equivalently $\Z_12$.
\end{example}
\begin{definition}[Simple Groups]
    A group is \emph{simple} if it is nontrivial and has no proper nontrivial normal subgroups.
\end{definition}
\begin{remark}
    The alternating group $A_n$ is simple for $n\geq 5.$
\end{remark}
\begin{theorem}
    Let $\phi\colon G\to G'$ be a group homomorphism. If $N \vartriangleleft G$, then $\phi[N] \vartriangleleft \phi[G].$ Also, if $N'$ is a normal subgroup of $\phi[G]$, then $\phi^{-1}[N'] \vartriangleleft G$. Note that $\phi[N]$ may not be normal in $G'$.
\end{theorem}
\begin{definition}[Maximal Normal Subgroup of a Group $G$]
    A \emph{maximal normal subgroup of a group $G$} is a normal subgroup $M$ not equal to $G$ such taht there is no proper normal subgroup $N$ of $G$ properly containing $M$.
\end{definition}
\begin{theorem}
    $M$ is a \emph{maximal normal subgroup} of $G$ if and only if $G/M$ is simple.
\end{theorem}
\begin{proof}
    $\implies\colon$Let $M$ be a maximal normal subgroup of $G$. Take the canonical homomorphism $\gamma\colon G\to G/M$ Now, $\gamma^{-1}$ of any nontrivial proper normal subgroup of $G/M$ is a proper normal subgroup of $G$ properly containing $M$. But $M$ is maximal so this isn't possible. So $G/M$ is simple.

    $\impliedby\colon$If $N$ is a normal subgroup of $G$ properly containing $M$, then $\gamma[N]$ is normal in $G/M$. If $N\neq G$, then $\gamma[N]\neq G/M$ and $\gamma[N]\neq\{M\}.$ If $G/M$ is simple, no such $\gamma[N]$ and thus no such $N$ can exist so $M$ is maximal.
\end{proof}
\begin{definition}[Center of $G$]
    Every nonabelian group has a \emph{center $Z(G)$} such that $$Z(G) = \{z \in G \mid zg =g z \forall g \in G\}.$$ The center always contains the identity, but if it only contains this then it is trivial.
\end{definition}
\begin{definition}[Commutator]
    To abelianze $G$, we will find all elements such that $ab=ba$, or that $aba^{-1}b^{-1}=e$. This element is a \emph{commutator of the group}.
\end{definition}
\begin{theorem}
    Let $G$ be a group. The set of all commutators $aba^{-1}b^{-1}$ for $a,b \in G$ generates the \emph{commutator subgroup} $C$ of $G$. This is a normal subgroup of $G$. Furthermore, if $N$ is a normal subgroup of $G$, then $G/N$ is abelian if and only if $C \leq N$.
\end{theorem}
\begin{proof}
    The commutators surely generate a subgroup $C$ from $e$, inverses, and closure. Now, for any $x \in C$, and any $g \in G$, $x = cdc^{-1}d^{-1}$ for some $c,d \in G$ such that $g^{-1}xg = (g^{-1}cdc^{-1})(e)(d^{-1}g)$. This becomes $(g^{-1}cdc^{-1})(gd^{-1}g^{-1})(d^{-1}g) = [(g^{-1}c)d(g^{-1}c)^{-1}d^{-1}][dg^{-1}d^{-1}g] \in C$ so $C \vartriangleleft G$. 
    
    Next, if $N \vartriangleleft G$, then $\implies\colon$If $G/N$ abelian, $(a^{-1}N)(b^{-1}N)=(b^{-1}N)(a^{-1}N)$ so $aba^{-1}b^{-1}N=N$ so $aba^{-1}b^{-1}\in N \implies C \leq N$. 
    
    $\impliedby\colon$If $C\leq N$, then $(aN)(bN) = abN = ab(b^{-1}a^{-1}ba)N = baN = (bN)(aN)$.
\end{proof}

\section{Group Action on a Set}

\begin{definition}[Group Action]
    Let $X$ be a set and $G$ a group. An \emph{action of $G$ on $X$} is a map $*\colon G\times X\to X$ such that: \begin{enumerate}
        \item $ex = x$ for all $x \in X$
        \item $(g_1g_2)(x) = g_1(g_2x)$ for all $x \in X$ and all $g_1, g_2 \in G$.
    \end{enumerate}
    In this case, we call $X$ a \emph{$G$-set}.
\end{definition}
\begin{theorem}
    Let $X$ be a $G$-set. For each $g\in G$, the function $\sigma_g\colon X\to X$ defined by $\sigma_g(x)=gx$ for $x \in X$ is a permutation of $X$. Also, the map $\phi\colon G\to S_x$ defined by $\phi(g) = \sigma_g$ is a homomorphism with the property that $\phi(g)(x)=gx$.
\end{theorem}
\begin{proof}
    To show $\sigma_g$ is a permutation of $X$, we must show it is bijective. (i) If $\sigma_g(x_1) = \sigma_g(x_2)$, then $gx_1=gx_2$ so $g^{-1}(gx_1)=g^{-1}(gx_2)$ so $(g^{-1}g)(x_1)=(g^{-1}g)(x_2)$ from the second condition of group actions so $e(x_1)=e(x_2)$ so $x_1=x_2$ from the first condition. Next, for any $x \in X$, $\sigma_g(g^{-1}x)=g(g^{-1})x=(gg^{-1})x=ex=x$ so $\sigma_g$ is onto and 1-1 making it a permutation.

    Next, $\phi\colon G\to S_X$ defined by $\phi(g) = \sigma_g$ is a homomorphism because $\sigma(g_1g_2)(x) = (g_1g_2)x=g_1(g_2x)=g_1\sigma_{g_2}(x)=\sigma_{g_1}(\sigma_{g_2}(x))=(\sigma_{g_1}\circ\sigma_{g_2})(x) = (\sigma_{g_1}\sigma_{g_2})(x)=(\phi(g_1)\phi(g_2))(x)$.
\end{proof}
\begin{definition}[Acting Faithfully, Transitive]
    Note that the subset of $G$ leaving each element of $X$ fixed is a normal subgroup $N$ of $G$. We say $G$ \emph{acts faithfully} on $X$ if $N = \{e\}$.

    We say $G$ is transitive on a $G$-set $X$ if and only if the subgroup $\phi[G]$ of $S_X$ is transitive on $X$, that is, if for each $x_1, x_2 \in X$, there exists some $g \in G$ so that $gx_1 = x_2$.
\end{definition}
\begin{remark}
    Every group is itself a $G-set$.
\end{remark}
\begin{theorem}
    Let $X$ be a $G$-set. Then $G_x$ is a subgroup of $G$ for each $x \in X$.
\end{theorem}
\begin{note}
    As notation, for $G$-set $X$ with $x \in X, g \in G$, we say $$X_g=\{x \in X\mid gx =x\}\quad\text{and}\quad G_x=\{g \in G\mid gx=x\}.$$
\end{note}
\begin{proof}
    Let $x \in X$, $g_1, g_2 \in G_x$. So $g_1x=g_2x=x$ so $g_1x=x=g_2x$. Thus, $(g_1g_2)x=g_1(g_2x)=g_1x=x$ so $g_1, g_2 \in G_x$ and $G_x$ closed under the induced operation of $G$. Clearly, $e \in G_x$. And $g \in G_x$ implies $x = ex = (g^{-1}g)x = g^{-1}(gx) = g^{-1}x$ so $g^{-1} \in G_x$. Thus $G_x \leq G$.
\end{proof}
\begin{definition}[Isotropy Subgroup]
    Let $X$ be a $G$-set and $x \in X$. The subgroup $G_x$ is the \emph{isotropy subgroup of $x$}.
\end{definition}
\begin{theorem}
    Let $X$ be a $G$-set. For $x_1, x_2 \in X$, say $x_1\sim x_2$ if and only if there exists $g \in G$ so $gx_1 = x_2$. $\sim$ is an equivalence relation on $X$.
\end{theorem}
\begin{proof}
    (i)For any $x \in X$, $ex = x$ so $x\sim x$. (ii) For any $x_1, x_2 \in X$, if $x_1 \sim x_2$, then $gx_1 = x_2$ so $g^{-1}x_2=g^{-1}(gx_1)=(g^{-1}g)(x_1)=ex_1=x_1$ so $x_2 \sim x_1$. (iii) Last, if $x_1\sim x_2, x_2 \sim x_3$, then $g_1x_1=x_2, g_2x_2=x_3$ for some $g_1, g_2 \in G$ so $(g_2g_1)(x_1) = g_2(g_1x_1) = g_2(x_2) = x_3$ so $x_1 \sim x_3$.
\end{proof}
\begin{definition}[Orbit of $x$]
    Let $X$ be a $G$-set. Each cell in the partition of the equivalence relation is described as a \emph{orbit in $X$ under $G$.} For $x \in X$, the cell containing $x$ is the \emph{orbit of $x$}. This is $Gx$.  
\end{definition}
\begin{theorem}
    Let $X$ be a $G$-set, $x \in X$. Then $\left|Gx\right| = (G\colon G_x)$. If $\left|G\right|$ is finite, then $|Gx|$ is a divisor of $|G|$.
\end{theorem}
\begin{proof}
    Let's define the 1-1 map $\psi$ from $G_X$ onto the collection of left cosets of $G_x$ in $G$. Let $x_1 \in Gx$. Then, there exists $g_1 \in G$ so $g_1x = x_1$. Say $\psi(x_1)$ is the left coset $g_1G_x$ of $G_x$. To show this is well defined, if $g_1'x = x_1$, then $g_1x = g_1'x$ so $g_1^{-1}(g_1x)=g_1^{-1}(g_1'x)$ implying $x = g(g_1^{-1}g_1')x$ so $g_1^{-1}g_1' \in G_x$ so $g_1' \in g_1G_x$ and $g_1G_x = g_1'G_x$.

    To show $\psi$ is one-one, $x_1,x_2 \in Gx$ gives $\psi(x_1)=\psi(x_2)$ so there exists $g_1, g_2 \in G$ so $x_1 = g_1x, x_2 = g_2x$ where $g_2 \in g_1Gx$ giving $g_2 = g_1g$ for some $g \in G_x$. Thus, $x_2 = g_2x = g_1(gx) = g_1x = x_1$. 

    To show it's onto, for any left coset of $G_x$ $g_1G_x$ in $G$, if $g_1x=x_1$ then $g_1G_x = \psi(x_1)$. This map is bijective so $|Gx| = (G\colon G_x)$. If $|G|$ finite, then clearly, $|Gx|$ divides $|G|$.
\end{proof}