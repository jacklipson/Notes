\chapter{Extension Fields}

\section{Introduction to Extension Fields}

\begin{definition}[Extension Field]
    A field $E$ is an \emph{extension field of a field $F$} if $F \leq E.$ For instance, we can write a \emph{tower of fields} as $\Q \leq \R \leq \C$ and $F \leq F(x), F(y) \leq F(x,y).$
\end{definition}
\begin{theorem}[Kronecker's Theorem]
    Let $F$ be a field and $f(x)$ be some nonconstant polynomial in $F[x].$ Then, there exists some extension field $E$ of $F$ and an $\alpha \in E$ where $f(\alpha) = 0$.
\end{theorem}
\begin{proof}
    By a prior theorem, $f(x)$ has some factorization in $F[x]$ into irreducible polynomials over $F.$ Say $p(x)$ is one such irreducible polynomial. It is sufficient to find an extension field $E$ of $F$ containing an element $\alpha$ so $p(\alpha) = 0.$ By an earlier theorem, $\langle p(x) \rangle$ is a maximal ideal in $F[x]$ implying $F[x]/\langle p(x) \rangle$ is a field. We can naturally define $\psi \colon F \to F[x]/\langle p(x) \rangle$ where $\psi(a) = a + \langle p(x) \rangle$ for $a \in F$. This is injective as $a + \langle p(x) \rangle = b + \langle p(x) \rangle, a,b \in F$ implies $(a-b) \in \langle p(x) \rangle$ so $a-b$ is a multiple of $p(x)$ of degree $\geq 1$ so $a - b = 0$ so $a = b.$ $\psi$ is easily a homomorphism which maps onto a subfield of $F[x]/\langle p(x) \rangle.$ We can thus identitfy $F$ with $\{a + \langle p(x) \rangle \mid a \in F\}$ so $E = F[x]/\langle p(x) \rangle$ is an extension field of $F$. 
    
    We're left to show $E$ has some zero of $p(x)$ which we can do via $\alpha = x + \langle p(x) \rangle, \alpha \in E$ so $\phi_\alpha\colon F[x] \to E$ by a previous theorem gives $p(x) = a_0 + a_1x + \cdots + a_nx^n, a_i \in F$ so $\phi_\alpha(p(x)) = a_0 + a_1(x + \langle p(x) \rangle) + \cdots + a_n(x + \langle p(x) \rangle)^n$ in $E$. But, we can compute via representatives and $x$ is a representative so $p(\alpha) = p(x) + \langle p(x) \rangle = \langle p(x) \rangle = 0$ so there exists some $\alpha \in E$ such that $p(\alpha) = 0$ and therefore $f(\alpha) = 0.$
\end{proof}
\begin{example}
    Let $F = \R$ and $f(x) = x^2+1$ which is clearly irreducible over $\R$ such that $\langle x^2 + 1 \rangle$ is a maximal ideal in $\R[x]$ so $\R[x]/\langle x^2 + 1 \rangle$ is a field. Identifying $r \in \R$ with $r + \langle x^2 + 1 \rangle$ lets us view $\R$ as a subfield of $\R[x] / \langle x^2 + 1 \rangle.$ Now, $\alpha = x + \langle x^2 + 1 \rangle$ so $\alpha^2 + 1 = (x + \langle x^2 + 1 \rangle)^2 + (1+\langle x^2 + 1 \rangle) = (x^2+1) + \langle x^2 + 1 \rangle = 0$ so $\alpha$ is a zero of $x^2 + 1.$
\end{example}
\begin{definition}[Algebraic + Transcendental]
    An element $\alpha$ of an extension field $E$ of a field $F$ is \emph{algebraic over $F$} if $f(\alpha) = 0$ for some nonzero $f(x) \in F[x]$. If $\alpha$ isn't, then it is \emph{transcendental over $F$}.
\end{definition}
\begin{example}
    $\sqrt{2}$ is an algebraic number over $\Q$ because it is a zero of $x^2-2$ while $i$ is also an algebraic element over $\Q$ because it is a zero of $x^2 + 1$ inside extension field $\C$.
\end{example}
\begin{example}
    The real number $\pi$ is transcendental over $\Q$ however $\pi$ is algebraic over $\R$ as it a zero of $(x-\pi) \in \R[x]$.
\end{example}
\begin{theorem}
    Given extension field $E$ of field $F$ and $\alpha \in E,$ let $\phi_\alpha\colon F[x] \to E$ be the evaluation homomorphism so $\phi_\alpha(a) = a$ for $a \in F$ and $\phi_\alpha(x) = \alpha.$ Thus, $\alpha$ is transecendental over $F$ iff $\phi_\alpha$ gives an isomorphism of $F[x]$ with a subdomain of $E$, that is iff $\phi_\alpha$ injective.
\end{theorem}
\begin{proof}
    The element $\alpha$ is transcendental over $F$ if and only if $f(\alpha) \neq 0$ for all nonzero $f(x) \in F[x]$ which is true iff (by definition), $\phi_\alpha(f(x)) \neq 0$ for all nonzero $f(x)$ which is true iff $\Ker\phi_\alpha = \{0\}$ iff $\phi_\alpha$ is injective.
\end{proof}
\begin{theorem}
    Let $E$ be an extenson field of $F$ with $\alpha \in E$ algebraic over $F$. Then, there is an irreducible polynomial $p(x) \in F[x]$ so $p(\alpha) = 0$. This polynomial is uniquely determined up to a constant factor and is a polynomial of minimal degree $\geq 1$ having $\alpha$ as a zero. If $f(\alpha) = 0$ for some $f(x) \in F[x]$ for $f(x) \neq 0$, then $p(x) \mid f(x).$
\end{theorem}
\begin{proof}
    Given evaluation homomorphism $\phi_\alpha$ of $F[x]$ into $E$, its kernel is an ideal and by a previous theorem, must be a principal ideal generated by some $p(x) \in F[x]$ implying $\langle p(x) \rangle$ consists precisely of those elements of $F[x]$ having $\alpha$ as a zero. So, if some $f(x) \neq 0$ and $f(\alpha) = 0$, then $f(x) \in \langle p(x) \rangle$ so $p(x) \mid f(x)$ making $p(x)$ a polynomial of minimal degree $\geq 1$ with zero $\alpha$ and any other polynomial of the same degree of form $(a)p(x), a \in F$. Now, to show $p(x)$ is irreducible, if $p(x) = r(x)s(x)$ were a possible factorization into polynomials of lower degree, then $p(\alpha)$ implies either $r(\alpha)$ or $s(\alpha)$ is 0 contradicting the fact $p(x)$ is of minimal degree $\geq 1$ with $p(\alpha) = 0$. So $p(x)$ is irreducible.  
\end{proof}
\begin{definition}[Monic Polynomial]
    A \emph{monic polynomial} is one with leading coefficient 1.
\end{definition}
\begin{definition}[Irreducible Polynomial for $\alpha$ over $F$]
    Given extension field $E$ of $F$ with $\alpha \in E$ algebraic over $F$, the unique monic polynomial $p(x)$ is the \emph{irreducible polynomial for $\alpha$ over $F$}, denoted $\text{irr}(\alpha, F)$ with degree of $\alpha$ over $F$ denoted $\text{deg}(\alpha,F).$
\end{definition}
\begin{example}
    $\text{irr}(\sqrt{2},\Q) = x^2-2$ is  degree 2 of $\alpha$ over $\Q.$
\end{example}
\begin{remark}
    With extension field $E$ of a field $F$ and $\alpha \in E$ and evaluation homomorphism $\phi_\alpha(a) = a$ for $a \in F$ and $\phi_\alpha(x) = \alpha$, there are two possible cases:
    \begin{enumerate}[label = \textbf{Case \Roman*}, leftmargin=*]
        \item If $\alpha$ is \emph{algebraic over $F$}, then the kernel of $\phi_\alpha$ is $\langle \text{irr}(\alpha, F)\rangle$ and therefore a maximal ideal of $F[x]$. Also, $F[x] / \langle \text{irr}(\alpha, F)\rangle$ is a field and isomorphic to the image $\phi_\alpha[F[x]]$ in $E$, making $\phi_\alpha[F[x]]$ of $E$ the smallest subfield of $E$ containg $F$ and $\alpha$, denoted $F(\alpha).$
        \item If $\alpha$ is \emph{algebraic over $F$}, then $\phi_\alpha$ gives an isomorphism of $F[x]$ with a subdomain of $E$. Thus, $\phi_\alpha[F[x]]$ is not a field, but instead an integral domain denoted by $F[\alpha]$. Consequently, $E$ contains a field of quotients of $F[\alpha]$ which is the smallest subfield of $E$ containing $F$ and $\alpha$ which we denote $F(\alpha).$
    \end{enumerate}
\end{remark}
\begin{remark}
    Since $\pi$ is transcendental over $\Q$, the field $\Q(\pi)$ is isomorphic to the field $\Q(x)$ of rational functions over $\Q[x]$.
\end{remark}
\begin{definition}[Simple Extension]
    An extension field $E$ of a field $F$ is a \emph{simple extension} of $F$ if $E = F(\alpha)$ for some $\alpha \in E$.
\end{definition}
\begin{theorem}
    With simple extension $E = F(\alpha)$ of a field $F$ and $\alpha$ algebraic over $F$, if the degree of $\text{irr}(\alpha,F) \geq 1$, then every element $\beta$ of $E$ can be uniquely expressed as $\beta = \Sigma_{i=0}^{n-1}b_i\alpha^{n-1}$ for $b_i \in F$.
\end{theorem}
\begin{proof}
    For the usual evaluation homomorphism $\phi_\alpha$, each element $F(\alpha) = \phi_\alpha[F[x]]$ is of the form $\phi_\alpha(f(x)) = f(\alpha)$, a formal polynomial in $\alpha$ with coefficients in $F$ so $\text{irr}(\alpha, F) = p(x) = x^n + \cdots + a_0$ so $p(\alpha) = 0$ means $\alpha^n = -a_{n-1}\alpha^{n-1} - \cdots - a_0$. This allows us to express any monomial $\alpha^m, m \geq n$ in terms of powers of $\alpha$ less than $n$, i.e. $\alpha^{n+1} = \alpha\alpha^n.$ Thus, if $\beta \in F(\alpha)$, $\beta = b_0 + b_1\alpha + \cdots + b_{n-1}\alpha^{n-1}$. For uniqueness, $b_0 + b_1\alpha + \cdots + b_{n-1}\alpha^{n-1} = b_0' + b_1'\alpha + \cdots + b_{n-1}'\alpha^{n-1}$ for $b_i, b_i' \in F$ implies $g(x) = (b_0 - b_0') + (b_1-b_1')x + \cdots + (b_{n-1} - b_{n-1}')x^{n-1} \in F[x]$ and $g(\alpha) = 0$. Also, the degree is less than the degree of $\text{irr}(\alpha,F)$ so because $\text{irr}(\alpha, F)$ is a nonzero polynomial of minimal degree with $\alpha$ as a zero, we must have $g(x) = 0$ so $b_i = b_i'$ proving uniqueness.
\end{proof}
\begin{example}
    The polynomial $p(x) = x^2+x+1$ in $\Z_2[x]$ is irreducible over $\Z_2$ since neither 0 nor 1 is a zero however we know there is an extension field $E$ containg a zero $\alpha$ of $x^2+x+1$. Specifically, $\Z_2(\alpha)$ has elements $0 + 0\alpha, 1 + 0\alpha, 0 + 1\alpha, 1 + 1\alpha$ giving us a new finite field of 4 elements. This gives us $(1+\alpha)^2 = \alpha$ because $\alpha^2 = \alpha + 1$. Thus, $\R[x]/\langle x^2 + 1 \rangle$ is isomorphic to the field $\C$ because we can view $\R[x] / \langle x^2 + 1 \rangle$ as an extension field of $\R = \R(\alpha)$. Because $\alpha^2 + 1 =0$ for some, we see $\alpha$ plays the role if $i \in \C$ and $a + b\alpha$ plays the role of $a + bi$ making $\R(\alpha) \sim \C$.
\end{example}


% \section{Vector Spaces}

% \section{Algebraic Extensions}