\chapter{Introduction to Groups}

\section{Binary Operations}
    
\begin{definition}[Binary Operation]
    A \textit{binary operation} \textbf{*} on a set $S$ is a rule that assigns to each ordered pair $(a,b)$ of elements of $S$ another element of $S$ generally denoted $a*b$ or formally $*(a,b)$. To be \emph{well-defined}, $*$ must assign a value to every possible $a*b$.
\end{definition}
\begin{definition}[Closure under $*$]
    A set $S$ is \emph{closed under $*$} if for all $a,b \in S$, $a*b \in S$. If a subset $H$ of $S$ is also closed under $*$, this is referred to as the \emph{induced operation} $*$ on $H$.
\end{definition}
\begin{definition}[Commutative Operation]
    A binary operation $*$ on a set $S$ is \textit{commutative} iff $a*b = b*a$ for all $a,b \in S$.
\end{definition}
\begin{definition}[Associative operation]
    A binary operation $*$ on a set $S$ is \textit{associative} iff $(a*b)*c = a*(b*c)$ for all $a,b,c \in S$.
\end{definition}
\begin{note}
    Associativty of function compostion follows.
\end{note}
\begin{remark}
    A binary operation on a set, typically finite, can be represented as follows:
    \[ 
        \renewcommand\arraystretch{1.2}
        \begin{tabu}{c|[2pt]c|c|c}
            * & a & b & c \\\tabucline[2pt]{-}
            a & b & b & b \\\hline
            b & a & c & b \\\hline
            c & c & b & a
        \end{tabu}
    \]
\end{remark}

\section{Groups}

\begin{definition}[Group]
    A group $\langle G, * \rangle$ is a set $G$ combined with a binary operation $*$ on $G$ which satisfies the following axioms:
    \begin{enumerate}[label = ($\mathscr{G}_{\arabic*}$)]
        \item $*$ is associative.
        \item There exists a \textbf{unique} \emph{identity} element \textit{e} on $G$ s.t. $e*x = x*e$ for all $x \in G$. 
        \item For each $a \in G$, there exists an $a' \in G$ s.t. $a'*a = a*a'=e$. This $a'$ is called the \textit{inverse} of $a$ with respect to the operation $*$.
        \item \textit{(optional if part of binary operation definition)} $G$ is closed under $*$.
    \end{enumerate}
\end{definition}
\begin{theorem}[Left/Right Cancellation]
    If $G$ is a group with binary operation $*$, then the \emph{left and right} cancellation laws hold s.t. $a*b=a*c \implies b=c$ and $b*a=c*a \implies b=c$ for all $a,b,c \in G$.
\end{theorem}
\begin{proof}
    The right cancellation proof is identical to that below.
    \begin{align*}
        a*b &= a*c & \text{$\because$ by supposition}\\
        a'*(a*b) &= a'*(a*c) & \text{$\because$ inverse axiom}.\\
        (a'*a)*b &= (a'*a)*c & \text{$\because$ associativity axiom} \\
        e*b &= e*c & \text{$\because$ inverse axiom} \\
        b &= c & \text{$\qed$ identity axiom}
    \end{align*}
\end{proof}
\begin{theorem}
    Trivially, in a group $G$, $(ab)' = b'a'$ for all $a,b \in G$.
\end{theorem}
\begin{remark}
    Note that the solutions $x,y$ to $a*x=b$ and $y*a=b$ have unique solutions in $G$ for any $a,b \in G$. Similarly, $e$ is unique.
\end{remark}
\begin{note}[Idempotent for $*$]
    An element $x$ of $S$ is \emph{idempotent for $*$} if $x*x=x$. This is always in the identity element.
\end{note}
\begin{definition}[Abelian Group]
    A group $G$ is \textit{abelian} if its binary operation is commutative. 
\end{definition}
\begin{definition}[Roots of Unity]
    Call the elements of the set $U_n \coloneq \{ z \in \C \mid z^n = 1\}$ the $n^{th}$ roots of unity, usually listed as 1 = $\zeta^0, \zeta^1, \zeta^2, \ldots, \zeta^{n-1}$.
\end{definition}
\begin{remark}
    Let the unit circle $U \coloneq \{z \in \C \mid |z| = 1\}$. Clearly, for any $z_1, z_2 \in U$, $|z_1z_2| = |z_1||z_2| = 1$ such that $z_1z_2 \in U$ implying U is closed under $\cdot$. Note then that $\langle U, \cdot \rangle \simeq \langle R_{2\pi}, +_{2\pi} \rangle$. Similarly, $\langle U_n, \cdot \rangle \simeq \langle \Z_n, +_n \rangle \text{ for } n \in \Z^+$.
\end{remark}
\begin{definition}[Addition Modulo n]
    We respecitvely write $\Z_n$ and $\R_c$ to denote $[0, 1, \ldots, n-1]$ and $[0, c]$. Addition modulo $n/c$ is written $+_n$ or ${+_c}$.
\end{definition}

\section{Isomorphic Binary Structures}

\begin{definition}[Binary Algebraic Structures]
    For two \emph{binary algebraic structures} $\langle S, * \rangle$ and $\langle S', *' \rangle$ to be structurally alike, we would need a one-to-one correspondence between the elements $x \in S$ and $x' \in S'$ s.t. if $x \leftrightarrow x'$ and $y \leftrightarrow y'$ then $x*y \leftrightarrow x'*'y'$.
\end{definition}
\begin{remark}[Homomorphism Property]
    This last condition is called the \emph{homorphism property}. If the function $\phi$ is NOT one-to-one, it is a homormorphism only. 
\end{remark}
\begin{definition}[Isomorphism]
    An \emph{isomorphism} of $S$ with $S'$ is a one-to-one function $\phi$ mapping $S$ onto $S'$ such that $\phi(x*y) = \phi(x) *' \phi(y)$ for all $x,y \in S$. 

    If such a map exists, $S$ and $S'$ are called \emph{isomorphic binary structures} denoted $S \simeq S'$.
\end{definition}
\begin{note}[Show Binary Algebraic Structures are Isomorphic]
    \leavevmode
    \begin{enumerate}[label=(Step \arabic*), leftmargin=*]
        \item Define the function $\phi$ which defines $\phi(s)$ for all $s \in S$ and gives the isomorphism from $S \to S'$.
        \item Show $\phi$ is one-to-one.
        \item Show $\phi$ is onto.
        \item Show $\phi(x*y) = \phi(x)*'\phi(y)$ for all $x,y \in S$.
    \end{enumerate}
\end{note}
\begin{example}
    Take the isomorphism $\phi\colon\R\to\R^+\colon x\longmapsto e^x$ from $\langle \R, +\rangle$ to $\langle \R^+, \cdot \rangle$. Clearly, $\forall x \in \R, \phi(x) \in \R^+$ and $\phi$ is bijective. Last, for $x,y \in \R$, $\phi(x + y) = e^{x*y} = e^xe^y = \phi(x)\cdot\phi(y)$.
\end{example}
\begin{definition}[Structural Property]
    A structural property is any property of a binary structure that is invariant to any isomorphic structure. These, like cardinality, are used to show no such isomorphism exists between structures.
\end{definition}
\begin{example}
    Although $\langle \Q, + \rangle$ and $\langle \Z, + \rangle$ both have cardinality $\aleph_0$ and have many one-to-one functions between them, the equation $x+x = c$ has a solution $x \in Q$ for all $c \in \Q$, but this is not true for $\Z$ if, say, $c=3$. This structural propery distinguishes these binary structures and thus they are not isomorphic under the usual addition.
\end{example}
\begin{theorem}
    Suppose $\langle S, * \rangle$ has an identity element $e$ for $*$. If $\phi \colon S \to S'$ is an isomorphism to $\langle S', *' \rangle$ then $\phi(e)$ is an identity element for $*'$ on $S'$.
\end{theorem}
\begin{proof}
    Because an isomorphism exists from $S \to S'$, for any element $s' \in S'$, there exists exactly one element $s \in S$ s.t. $\phi(s) = s'$. By the definition of an isomorphism $s' = \phi(s) = \phi(s * e) = \phi(s) *' \phi(e) = s' *' \phi(e)$ for an arbitary element $s'$ of $S$. This implies $\phi(e)$ is the identity element for $S'$.
\end{proof}

\section{More on Groups and Subgroups}

\begin{definition}[Semigroup]
    A semigroup is an algebraic structure combining a set with an associative binary oxperation.
\end{definition}
\begin{definition}[Monoid]
    A monoid is a semigroup that has an identity element corresponding to its binary operation. 
\end{definition}
\begin{definition}[Subgroup]
    If a subset $H$ of a group $G$ is closed under the binary operation of $G$ and is itself a group, $H$ is a \emph{subgroup} of $G$. This is denoted $H \leq G$. $H < G \implies H \neq G$.
\end{definition}
\begin{example}
    \ \group{\Z}{+} < \group{\R}{+}, but \group{\Q}{\cdot} is \emph{not} a subgroup of \group{\R}{-}.
\end{example}
\begin{definition}[Proper and trivial subgroups]
    If $G$ is a group, the subgroup consisting of $G$ itself is the \emph{improper subgroup} of $G$. All other subgroups are \emph{proper subgroups}. The subgroup $\{e\}$ is the \emph{trivial subgroup} of $G$ and all other subgroups are nontrivial.
\end{definition}
\begin{theorem}
    A subset $H$ of a group $G$ is a subgroup of $G$ if and only if:
    \begin{enumerate}
        \item $H$ is closed under the binary operation of $G$.
        \item the identity $e$ of $G$ is in $H$.
        \item for all $a \in H$, $a^{-1} \in H$ also.
    \end{enumerate}
\end{theorem}
\begin{proof}
    $\implies:$ Let $H$ be a subgroup of $G$. By definition, $H$ is closed under $G$'s binary operation (1). $H$ must have an identity element because it is a group. Because $a*x=a$ and $y*a=a$ have unique solutions, $H$'s identity element must be the same in $H$ group as $G$ group (2). (3) is trivial because $H$ is a group.

    $\impliedby:$ Let (1), (2), (3) be true. Then $H$ has a unique identity element on its binary operation ($\mathscr{G}_2$), each element of $H$ has a unique inverse in $H$ ($\mathscr{G}_3$), and $H$ is closed under the binary operation of $G$ (\emph{optional} $\mathscr{G}_4$). To satisfy ($\mathscr{G}_1$), the binary operation on $H$ must be associative s.t., for all $a,b,c \in H$, $(ab)c = a(bc)$. But this is clearly holds in $G$ so ($\mathscr{G}_1$) is satisfied and $H$ is a subgroup of $G$.
\end{proof}
\begin{theorem}
    Let $G$ be a group and $a \in G$. Then $$H = \{a^n \mid n \in \Z\}$$ is a subgroup of $G$ and the \emph{smallest} subgroup of $G$ that contains $a$.
\end{theorem}
\begin{proof}
    
\end{proof}
\begin{remark}[Subgroup Diagrams]
    Lattice, or \emph{subgroup diagrams}, can be drawn such that lines run down from a group $G$ to a group $H$ if $H < G$. 
\end{remark}
\begin{example}
    Take two group structures of order 4: $\Z_4$ and the Klein 4-group \emph{Vierergruppe} defined as follows:

    \begin{center}
    $\Z_4:$
    \renewcommand\arraystretch{1.2}
    \begin{tabu}{c|[2pt]c|c|c|c}
        + & 0 & 1 & 2 & 3 \\\tabucline[2pt]{-}
        0 & 1 & 2 & 3 & 0 \\\hline
        1 & 2 & 3 & 0 & 1 \\\hline
        2 & 3 & 0 & 1 & 2 \\\hline
        3 & 0 & 1 & 2 & 3
    \end{tabu}
    \quad $V:$
    \begin{tabu}{c|[2pt]c|c|c|c}
        + & 0 & 1 & 2 & 3 \\\tabucline[2pt]{-}
        0 & 1 & 2 & 3 & 0 \\\hline
        1 & 2 & 3 & 0 & 1 \\\hline
        2 & 3 & 0 & 1 & 2 \\\hline
        3 & 0 & 1 & 2 & 3
    \end{tabu}
\end{center}
    We can map these as:
    \centering
    \begin{tikzpicture}[node distance=2cm]
        \node(Z4) {$Z_4$};
        \node(02) [below=.5cm of Z4] {$\{0,2\}$};
        \node(0) [below=.5cm of 02] {$\{0\}$};
        \draw (Z4) -- (02) -- (0);
    \end{tikzpicture}
    and
    \begin{tikzpicture}[node distance=2cm]
        \node(V) {$V$};
        \node(ea) [below left =.5cm and .5cm of V] {$\{e,a\}$};
        \node(eb) [below =.5cm of V] {$\{e,b\}$};
        \node(ec) [below right =.5cm and .5cm of V] {$\{e,c\}$};
        \node(e) [below =.5cm of eb] {$\{e\}$};
        \draw (V) -- (ea) -- (e);
        \draw (V) -- (eb) -- (e);
        \draw (V) -- (ec) -- (e);
    \end{tikzpicture}
    .
\end{example}