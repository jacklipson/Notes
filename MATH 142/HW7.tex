\subsection*{Page 91, Problem 1}
\vspace{15pt}
\begin{proof}
    \vspace{-10pt}
    Say $f\colon C \to C$ is a map that is NOT homeomorphic to the identity where $C$ is the unit circle. Suppose also $f(x) \neq -x$ for all $x \in C$. Thus, no antipodal points exist (taking points as (x,y) with additive inverse (-x, -y)) so no straight line between the identity and $f(x)$ pass through the origin. Therefore, via example 2 on page 89, we can define the homotopy from $f$ to $i$ $F\colon C\times I \to C$ as $$F(x,t) = \frac{(1-t)f(x)+tx}{||(1-t)f(x)+tx||}.$$ Thus, $f(x) \underset{F}{\cong} i$ which is a contradiction so there exists some point $x \in C$ so $f(x) = -x.$
\end{proof}

\subsection*{Page 95, Problem 13}
\vspace{15pt}
\begin{proof}
    \vspace{-10pt}
    Say $f\colon C \to C$ is a map that is NOT homeomorphic to the identity where $C$ is the unit circle. Suppose also $f(x) \neq -x$ for all $x \in C$. Thus, no antipodal points exist (taking points as (x,y) with additive inverse (-x, -y)) so no straight line between the identity and $f(x)$ pass through the origin. Therefore, via example 2 on page 89, we can define the homotopy from $f$ to $i$ $F\colon C\times I \to C$ as $$F(x,t) = \frac{(1-t)f(x)+tx}{||(1-t)f(x)+tx||}.$$ Thus, $f(x) \underset{F}{\cong} i$ which is a contradiction so there exists some point $x \in C$ so $f(x) = -x.$
\end{proof}
