\chapter{Introduction}

\section{Basic Definitions/Concepts}

\begin{definition}[Topological Space]
    A set $X$ \textbf{topological space} is a topological space if for each $x$ of $X$, there is a nonempty collection of subsets of $X$, called neighbourhoods of $x$, which satisfy the following axioms:
    \begin{enumerate}[(a)]
        \item $x$ lies in each of its neighbourhoods.
        \item The intersection of two neighbourhoods of $x$ is itself a neighbourhood of $x$.
        \item If $N$ is a neighbourhood of $x$ and if $U$ is a subset of $X$ which contains $N$, then $U$ is a neighbourhood of $x$.
        \item If $N$ is a neighbourhood of $x$, then we denote \textbf{the interior} of $N$ as the set $\interior{N}\coloneq\{z \in N \mid N \text{ is a neighbourhood of $z$}\}$. $\interior{N}$ is a neighbourhood of $x$.
    \end{enumerate}
    We say, if (a)-(d) are satisfied to each point $x \in X$, then there is a \textbf{topology} on the set $X$.
\end{definition}
\begin{definition}[Map]
    Let $X,Y$ be topological spaces. A function $f\colon X\to Y$ is \textbf{continuous} if for each point $x$ of $X$ and each neighbourhood $N$ of $f(x)$ in $Y$, the set $f^{-1}(N)$ is a neighbourhood of $x$ in $X$. Continuous functions are called \textbf{maps}.
\end{definition}
\begin{definition}[Homeomorphism]
    A function $h\colon X\to Y$ is a \textbf{homeomorphism} if it is one-one, onto, continuous, and has a continuous inverse. When such a function exists, $X$ and $Y$ are called \textbf{homeomorphic (or topologically equivalent) spaces}.
\end{definition}
\begin{definition}[Surface]
    A \textbf{surface} is a topological space in which each point has a neighbourhood homeomorphic to the plane, and for which any two distinct points possess disjoint neighbourhoods.
\end{definition}
\begin{definition}[Open]
    Let $X$ be a topological space and call a subset $O$ of $X$ \textbf{open} if it is a neighbourhood of each of its points.
\end{definition}
\begin{remark}
    From axiom (c), the union of any collection of open sets is open and from axiom (b) the intersection of any \emph{finite} number of open sets is open. Lastly, (d) shows that the interior of $N$ is an open set which lies inside $N$ and contains $x$. 
\end{remark}
\begin{definition}[New and Improved Topological Space]
    A topology on a set $X$ is a nonempty collection of subsets of $X$, which we call open sets, such that:
    \begin{enumerate}
        \item any union of open sets is itself open
        \item any finite intersection of open sets is open
        \item both the whole set $X$ and the empty set are open.
    \end{enumerate}
    Given a point $x$ of $X$, we shall call a subset $N$ of $X$ a \emph{neighbourhood} of $x$ if we can find an open set $O$ so $x \in O \subseteq N$. A set together with a topology on it is a topological space.
\end{definition}
\begin{proof}
    This set $X$ is a topological space because for each $x \in X$, $X$ is an open neighborhood of $x$ (a). This also confirms (c). If $N_1, N_2$ are neighbourhoods of $x$, we can find open sets $O_1, O_2$ so $x \in O_1 \subseteq N_1$ and $x \in O_2 \subseteq N_2$ such that $x \in O_1 \cap O_2 \subseteq N_1 \cap N_2$ Because $O_1 \cap O_2$ is open, $N_1 \cap N_2$ is a neighborhood of $x$ (b). If $N$ is a neighbourhood of $x$ then there is an open set $O \subseteq N$ so $x \in O$. By definition, $O$ is a neighborhood of each of its points. $\interior{N}$ is the set of all points $z$ that $N$ is a neighbourhood of. Clearly, then, $O$ is contained in $\interior{N}$. Thus, $\interior{N}$ is a neighborhood of $x$.
\end{proof}
\begin{definition}[Usual Topology on $\E^n$]
    A set $U$ is open if given $x \in U$, there exists $\epsilon \in R^+$ so the ball with centre $x$ and radius $\epsilon$ lies entirely in $U$.
\end{definition}
\begin{definition}[Subspace/Induced Topology]
    Given a topological space $X$ and a subset $Y$ of $X$, the open sets of the \textbf{subspace/induced} topology on $Y$ are simply the intersection of all the open sets of $X$ with $Y$.

    i.e. A subset $U$ of $Y$ is open in the subspace topology if there exists an open set $O$ of $X$ so $U = O\cap Y$.

    A subspace $Y$ of a topological space $X$ implies that $Y$ is a subset of $X$ with the subspace topology.
\end{definition}
\begin{definition}[Discrete Topology]
    The largest possible topology on a given set $X$ is the \textbf{discrete topology} wherein every subset of $X$ is an open set. If $X$ has the discrete topology, we call it a discrete space.
\end{definition}
\begin{example}
    If we take the set of points of $\E^n$ which have integer coordinates, and give it the subset topology, the result is a discrete space.
\end{example}
\begin{definition}["Larger" Topologies]
    If one topology contains all the open sets of another, we say it is \textbf{larger}.
\end{definition}
\begin{definition}[Closed]
    A subset of a topological space is closed if its complement is open.
\end{definition}
\begin{example}
    The following subsets of $\E^2$ are closed: the unit circle, the unit disk ($\{(x,y) \mid x^2+y^2 \leq 1\}$), $y=e^x$, and $\{(x,y) \mid x \geq y^2\}$.

    The set of all points $(x,y)$ where $x \geq 0, y > 0$ is neither open nor closed.

    The space $X$ of all points $(x,y)$ where $x\geq 1, y \leq -1$ with the topology induced from $\E^2$ is both open and closed in $X$ (and notably not open in $\E^2$).
\end{example}
\begin{remark}
    The intersection of any family of closed sets is closed. As is the union of any \emph{finite} family of closed sets.
\end{remark}
\begin{definition}[Limit Point]
    Let $A$ be a subset of a topological space $X$. A point $p \in X$ is a \textbf{limit point} (or accumulation point) of $A$ if every neighbourhood of $p$ contains at least one point of $A - \{p\}$.
\end{definition}
\begin{example}
    Give the set of all real numbers $X$ the \emph{finite-complement topology} where a set is open if its complement is finite or all of $X$. If we take $A$ to be an infinite subset of $X$, then every point of $X$ is a limit point of $A$. Conversely, a finite subset of $X$ has no limit points in this topology.
\end{example}
\begin{explanation}
    To be a neighbourhood $N$ of any $p \in X$, $N$ must be open implying its complement is either finite or $X$. If $N^C = X$, $N = \O$ so $N$ cannot be a neighbourhood of $p$ (this definition simply ensures $\O$ is open so this is indeed a topology). Thus, to be a neighbourhood, $N$ must be infinite with a finite complement. 
    
    If $A$ is an infinite subset of $X$, it must then share some infinite points with $N$ implying $N$ contains points of $A - \{p\}$. Because this is the case for all $N$ of $p$ and all $p \in X$, $p$ is a limit point of $A$.
    
    If $A$ is a finite subset, there exists neighbourhoods such that $N^C = A$ so every neighbourhood of $p$ certainly does not contain one point of $A - \{p\}$ implying no point of $X$ is a limit point of $A$.
\end{explanation}
\begin{theorem}
    A set is closed if and only if it contains all its limit points.
\end{theorem}
\begin{proof}
    $\implies$: If $A$ is closed, then its complement $X - A$ is open so $X-A$ is a neighbourhood of each of its points. Clearly, if a limit point $p$ of $A$ were in $X-A$ then $X-A$ must contain a point of $A - \{p\}$ of which there are none. So $A$ contains all its limit points.

    $\impliedby$: Suppose $A$ contains all its limit points. If $x \in X-A$, $x$ is not a limit point of $A$ so there exists a neighbourhood $N$ of $x$ which contains no point of $A$ implying $N \subseteq X-A$ such that $X-A$ is also a neighbourhood of $x$ for all $x \in X-A$ so $X-A$ is a neighbourhood of all of its points so it is open meaning $A$ is closed.
\end{proof}
\begin{definition}[Closure]
    The union of $A$ and all its limit points is called the \textbf{closure} of $A$ and is written $\close{A}$.
\end{definition}
\begin{theorem}
    The closure of $A$ is the smallest closed set containg $A$. i.e. the closure is the intersection of all closed sets containing $A$.
\end{theorem}
\begin{proof}
    The closure of $A$ is closed because if $x \in X-\close{A}$ then $x$ cannot be a limit point of $A$ so there exists an open neighbourhood $N$ of $x$ such that it contains no points of $A$. Because $N$ is an open set, it is a neighbourhood of all of its points so none of its points are limit points of $A$ either. Thus, $N \subseteq X-A$ so $X-A$ is a neighbourhood of $x$ so $X-A$ is a neighbourhood of each of its points so $X-\close{A}$ is open so $\close{A}$ is closed. Because $\close{A}$ is closed, contains $A$, and is contained in every closed set containing $A$, it must be the intersection of all such sets.

    \emph{NOTE: If we just said there exists a neighbourhood $N$ of $x$, this neighbourhood may contain a limit point of $A$ even if it does not contain a point of $A$. Thus, it is meaningful to prove none of its points can be limit points of $A$ by saying the neighbourhood is open.}
\end{proof}
\begin{corollary}
    A set is closed if and only if it is equal to its closure.
\end{corollary}
\begin{definition}[Dense]
    A set whose closure is the whole space is said to be \textbf{dense} in the space.
\end{definition}
\begin{definition}[Interior]
    The \textbf{interior} $\interior{A}$ of a set $A$ is the union of all open sets contained in $A$. A point $x$ lies in the interior of $A$ if and only if $A$ is a neighbourhood of $x$. Also, an open set is its own interior.
\end{definition}
\begin{example}
    In $\E^2$, denote the unit disk $D$ and the unit circle $C$. $D$'s interior is $D-C$ while $C$'s interior is $\O$. 
\end{example}
\begin{definition}[Frontier]
    The \textbf{frontier} of a set $A$ is the intersection of $\close{A}$ with $\close{X-A}$. This is equivalent to the points of $X$ neither in the interior of $A$ nor $X-A$.
\end{definition}
\begin{example}
    In $\E^2$, the unit disc $D$, its interior $\interior{D}$, and the unit circle $C$ all have the same frontier $C$.

    The froniter of the set of points in $\E^3$ which have rational coordinates is all of $\E^3$. In this case, the frontier is the whole space.
\end{example}
\begin{definition}[Base/Basis]
    Given a topology on a set $X$, a collection $\beta$ of open sets is called a \textbf{base/basis} for the topology if every open set is a union of members of $\beta$. Elements of $\beta$ are called \emph{basic open sets}.

    Equivalently, given any point $x \in X$ and a neighbourhood $N$ of $x$, there is always an element $B$ of $\beta$ so $x\in B\subseteq\beta$.
\end{definition}
\begin{theorem}
    Let $\beta$ be a nonempty collection of subsets of a set $X$. If the intersection of any finite number of members of $\beta$ is always in $\beta$, and if $\bigcup\beta=X$, then $\beta$ is a base for a topology on $X$.
\end{theorem}
\begin{proof}
    Let the collection of all possible unions of members of $\beta$ be open sets. This then immediately satisfies our new definition for a topological space.
\end{proof}