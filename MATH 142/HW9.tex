\subsection*{Page 109, Problem 25}
\vspace{15pt}
\begin{proof}
    \vspace{-10pt}
    Take the square $[0,1]\times[0,1]$ such that two opposite  edge pairs can be identified to create the torus $T^2$. Once the torus is wrapped together, imagine a puncture poked directly on the seam of the cylinder which made the torus but not on the seam of where the cylinder's boundary circles were identified, i.e., say $x =  1/2$ along the cylinder's length but half around $y=0$ and half around $y=1$. After being unwrapped, this will essentially have created a square with 2 small divets at the top and bottom at the halfway point along its width (like a fat 'H' shape). We can smoothly homotopy each divet taller and taller until we reach an $H$ shape with a skinny segment connecting it (think, something like $F((x,y),t) = \begin{cases}(x,y) \quad \text{for }x < 1/2 - \delta \vee x > 1/2 + \delta\\ (1-t)(x,y) + t(x,1/2)) \quad \text{for } |x - 1/2| < \delta \end{cases}$. Note that, still, this modified square is TOTALLY homotopically equivalent to the square which produces the punctured torus. So our 2 identifations will recreate the punctured torus.
    
    Applying the first identification connects the leftmost ends to each other and the right most ends to each other making 'o-o' where the two circles lie in a plane perpendicular to the segment connecting them. Doing the second identification identifies each circle together (imagine holding the connection of a pair of handcuffs down fixed and lifting up and gluing each circle together). This will form a new loop via the small segment. Because the line segment and each circle were only connected at a single point based on the above homotopy, just rotating one loop will form 2 circles with union of a single point. This completes our deformation retract.
\end{proof}


\subsection*{Page 109, Problem 26}
\vspace{15pt}
\begin{proof}
    \vspace{-10pt}
    \begin{enumerate}[label = (\alph*)]
        \item Take the boundary circle $C$ embedded in the möbius strip $S$. Any basepoint on the circle $C$ will have fundamental group $\Z$ representing the number and direction of winds around it. Therefore, we will take the homotopy class of a single $2\pi$ loop $\alpha$ to generate the group. 
        
        Now, take the rectangle $[0,10] \times [0,1]$ which, when its edges are inversely identified, creates the möbius strip. We can apply the continuous homotopy $F((x,y),t) = (1-t)(x,y) + t(x,1/2)$ to turn this pre-identification möbius strip into the boundary circle $C$ implying after identification, it will simply have the fundamental group $\Z$ as well, using the same basepoint as $C$. More visually, any jumble of a loop on the pre-identication strip can be straightened to the identity because the set is convex. But, post-identification, a loop can be formed by traveling straight from $(0,1/2$ along the $x$-axis until reaching $x=1$ which is the same point. I.e. following a path paralleling the edge of the mobius strip will eventually bring you to the same point. This implies the möbius strip has the same generator $\alpha$. Thus the \emph{identity group isomorphism} between cyclic groups $\pi_1(S^1) \to \pi_1(S^1)$ or really $\Z\to\Z$ satisfies our homomorphism. This final result is directly analogous to the solution for (c).

        \item Take the diagonal circle $C = \{(x,y) \in T^2 \mid x=y\}$ embedded in $S = S^1\times S^1$. To visualize $C$ with basepoint $(1/2,1/2)$, go back to the $[0,1]^2$ torus and draw the line $y = x.$ Clearly, after both identifications, this will be a loop as $(0,0) = (1,1).$ However, note that this loop, say $\gamma$, loops horizontally \emph{and} vertically. After the first identification, the line will travel from $(0,0)$ at its leftmost bottom point around the back of the cylinder up until the top of the cylinder at $x=1/2$ and then back down around the front to $(1,1) = (0,0)$. Therefore, after the final identification, it will form a loop that is both around the torus's perimeter ($x$-axis) and its girth ($y$-axis). As a result, performing this loop multiple times will perform both inseparable 'components' simultaneously that many times. $\pi_1(S^1 \times S^1) = \Z\times\Z$ is generated by these two loops – one around its ring (from $\alpha(t) = (1/2,t)$ on the square) and another around its outside $\beta(t) = (t, 1/2)$). These two loops generate one copy of $\Z$ that together form fundamental group $\Z^2$. Because each iteration of $C$'s generator always does $\alpha, \beta$ once each, we can form the homomorphism $\gamma_*\colon\pi_1(C,(1/2,1/2))\to\pi_1(S^1\times S^1,(1/2,1/2))$ defined via $\langle \gamma^n \rangle \mapsto \langle(\alpha^n, \beta^n) \rangle$ where $\pi_1(C) = \Z$ so really this function sends $k \in \Z$ to $(k,k) \in \Z^2$ which is indeed a homomorphism using addition.

        \item Take the boundary circle $S^{1}\times{0}$ embedded in the cylinder $S^1 \times [0,1]$. Once again, we can homotopy the cylinder via $F(x,y)$ where $x,y$ refers to the $[0,1]^2$ square which can have its top and bottom edges identified to create the cylinder. The continuous homotopy $F((x,y),t) = (1-t)(x,y) + t(0,y)$ tells us the cylinder is homotopically equivalent to the boundary circle $C$ or $S^1$. Both consequently have fundamental groups $\Z$ regardless of their basepoint and we can utilize the identity group isomorphism once more because the fundamental groups as well as the surfaces themselves are identical. For a more visual example, however, any loop winding around the surface of the cylinder will be slowly squashed via the homotopy until it is just looping around the circle. Because this homomorphism is from $\Z\to\Z$, addition in $\Z$ again clearly satisfies this homomorphism.
    \end{enumerate}
\end{proof}