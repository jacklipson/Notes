\chapter{Heat and the First Law of Thermodynamics}

\section{Heat as Energy Transfer}

\begin{remark}
    An 18-th century model pictured heatflow as movement of a fluid substance called \emph{caloric} which could never be detected. Today, heat, like work, represents a transfer of energy.
\end{remark}
\begin{definition}[Calorie]
    A \emph{calorie} (cal) is defined as the amount of heat necessary to raise the temperature of 1 gram of water by 1 Celsius degree. The more often used \emph{kilocalorie} (kcal), or \emph{\textbf{C}alorie} is 1000 calories. In the UK, heat is measured through \emph{British thermal units} (Btu) defined as the heat needed to raise the temperature of 1 lb of water by 1 F$\degree$. Gas companies use the \emph{therm} defined to be \qty{1e5}{Btu}.
\end{definition}
\begin{note}[Mechanical Equivalent of Heat]
    The mechanical equivalent oheat is known as \qty{4.186}{J} = \qty{1}{cal}. And, \qty{1}{Btu} = \qty{0.252}{kcal} = \qty{1056}{J}.
\end{note}
\begin{definition}[Heat]
    \emph{Heat} is energy transfered from one object to another because of a difference in temperature.
\end{definition}

\begin{definition}[Internal/Thermal Energy]
    The sum total of all the neergy of the molecules in an object is called its \emph{internal/thermal energy}. e.g. 2 objects of equal temperature have more internal temperature than just 1.

    For monatomic (one atom/molecule) gas, the internal energy $$E_{int} = N(\frac{1}{2}\bar{v^2}) = \frac{3}{2}NkT=\frac{3}{2}nRT.$$ If the gas molecules contain more than one atom, the rotaional and vibrational energy of the molecules must also be taken into account.
\end{definition}
\begin{remark}
    Internal energy of real gases, as opposed to ideal gases, also depends on pressure and volume (due to atomic potential energy). Similarly, the internal energy of liquids and solids includes electric potential energy associated with chemical bonds. 
\end{remark}

\section{Specific Heat}

\begin{definition}[Specific Heat]
    An amount of heat $Q$ put into an object of mass $m$ results, depending on its \emph{specific heat} $c$, in a temperature change of $\Delta T$ such that $$Q = mc\Delta T.$$
\end{definition}
\begin{definition}[Closed/Open/Isolated Systems]
    A \emph{closed system} is one for which no mass enters or leaves, but energy may be exchanged with the environment. Mass may enter or leave in an \emph{open system}. If no energy AND no mass passes its boundaries, a system is \emph{isolated}.
\end{definition}
\begin{remark}[Energy Conservation for Heat Transfer]
    Within an isolated system, we can write the energy conservation equation for heat transfer as $\Sigma Q = 0$.
\end{remark}
\begin{definition}[Calorimetry]
    \emph{Calorimetry} is a technique which quantitatively measures heat exchange using a \emph{calorimeter} that must be well insulated.
\end{definition}
\begin{definition}[Bomb Calorimeter]
    A \emph{bomb calorimeter} measures the thermal energy released when a substance burns to determine their Calorie content. In this calorimeter, a carefully weighed sample is placed with an excess of oxygen in the "bomb" that is then placed in the water of a calorimeter with a fine wire passing into the bomb that ignites the mixture.
\end{definition}

\section{Latent Heat}

\begin{definition}[Change of Phase]
    A certain amount of energy is required to change a material's phase.
\end{definition}
\begin{definition}[Heat of Fusion/Vaporization]
    The heat required to change \qty{1.0}{kg} of a substance from the solid to liquid state is called the \emph{heat of fusion}, denoted by $L_F$. The heat required for the change from liquid to the vapor phase is called the \emph{heat of vaporization} $L_V$. These are also called the \emph{latent heats} of substances.

    The heat involved in a change of phase is written as $Q = mL$ where $m$ is the mass of the substance.
\end{definition}
\begin{remark}
    These latent heats also refer to the amount of heat \emph{released} by a substance when it changes from gas to liquid or liquid to solid.
\end{remark}
\begin{remark}
    At the melting point, the latent heat of fusion does not increase the average kinetic energy (and thus the temperature) of the molecules in a solid, but instead overcomes the potential energy from the forces between molecules. This is the same for vaporization. However, vaporization requires a greater average distance between molecules such that the heat of vaporization is far greater than the heat of fusion for a given substance.
\end{remark}

\section{The First Law of Thermodynamics}

\begin{definition}[The First Law of Thermodynamics]
    Because heat is defined to be the transfer of energy due to a difference in temperature, let work be the transfer of energy NOT due to a temperature difference such that, as a general statement of the \emph{law of conservation of energy} $\Delta E_{int} = Q - W$.
\end{definition}
\begin{note}[Sign Conventions]
    Because $Q$ is the net heat \emph{added} to the system while $W$ is the net work done \emph{by} the system, we say the following:
    \begin{enumerate}
        \item Heat added is +
        \item Heat lost is -
        \item Work on system is -
        \item Work by system is +
    \end{enumerate}
\end{note}
\begin{remark}
    A full summary of the first law of thermodynamics would include kinetic energy $K$ and potential energy $U$ such that $$\Delta K + \Delta U + \Delta E_{int} = Q-W.$$
\end{remark}

\begin{definition}[Idealized Processes]
    An idealized process carried out at:
    \begin{enumerate}
        \item constant temperature ($\Delta T = 0$) is called \emph{isothermal}.
        \item constant heat ($Q = 0$) is called \emph{adiabatic}. (This is the case for well-insulated systems or very rapid processes.)
        \item constant pressure ($\Delta P = 0$) is called \emph{isobaric}.
        \item constant volume ($\Delta V = 0$) is called \emph{isovolumetric} (or isochoric).
    \end{enumerate}
\end{definition}
\begin{definition}[Isotherms]
    Curves on a PV diagram at different temperatures are called \emph{isotherms}.
\end{definition}
\begin{definition}[Quasistatically]
    A process done slowly enough such that a series of equilibrium states are essentially maintained is called \emph{quasistatically}.
\end{definition}
\begin{definition}[Heat Reservoir]
    A body whose mass is so large that, ideally, its temperature does not change significantly when heat is exchanged, is called a \emph{heat reservoir}.
\end{definition}
\begin{example}
    Suppose there is a gas of pressure $P$ inside a cylindrical container of area $A$ and length $\ell$ fitted with a movable piston.

    If our system is just the gas, as it exerts a force $F=PA$ on the piston, the work done by the gas is $dW = \vect{F}\cdot d\vect{\ell} = PAd\ell = PdV$ where $d\vect{l}$ is an infinitesimal displacement of the piston.

    If the gas was \emph{compressed} so $d\vect{\ell}$ would point into the gas, its volume would decrease and $dV<0$. Thus, the work done by the gas would be negative s.t. positive work is being done \emph{on} the gas. For a finite change in volume from $V_A$ to $V_B$, the work $W$ done by the gas will be $W = \int dW = \int_{V_A}^{V_B}PdV$.

    For an quasistatic isothermal expansion of an ideal gas, $P=nRT/V$ for constant $T$ tells us $W = nRT\ln\frac{V_B}{V_A}.$ This is the area under the curve between points $A$ and $B$ on a $PV$ diagram.

    We can replicate this same change of state from A to B for an ideal gas first isovolumetrically then isobarically. For an ideal gas undergoing an isovolumetric process where we lower the pressure from $P_A \to P_B$, we simply reduce the temperature by letting heat flow out ($dV = 0 \implies W = 0$ so no work is done). For an ideal gas undergoing an isobaric process taking the volume from $V_A \to V_B$, $P$ is constant at $P_B$ so $W = P_B\Delta V = P_B(V_B-V_A) = \frac{nRT_B}{V_B}(V_B-V_A) = nRT_B(1-\frac{V_A}{V_B})$. Thus, the total work from $A$ to $B$ in this case was $\frac{nRT_B}{V_B}(V_B-V_A)$ which is quantitatively different from $nRT\ln\frac{V_B}{V_A}$.
\end{example}
\begin{definition}[Free Expansion]
    \emph{Free expansion} is a type of adiabatic process when a gas is allowed to expand in volume without doing any work. Picture 2 well-insulated containers connected by a valve such taht one container has gas while the other is empty. When the valve is open, no heat flows or work is done because no objects are moved so $\Delta E_{int} = Q = W = 0$. $E_{int}$ only depends on $T$ so $\Delta T = 0$ as well.
\end{definition}

\section{Molar Specific Heats for Gases}

\begin{remark}
    Specific heats easily apply to solids and liquids. Gases depend much more on how the thermodynamic process is carried out however. Specific heats are given at constant volume $c_V$ and pressure $c_P$.
\end{remark}
\begin{definition}[Molar Specific Heats]
    Molar specific heats $C_V$ and $C_P$ denote the heat required to raise 1 mol of gas by 1 C$\degree$ at constant volume and pressure respectively. Thus,
    \begin{align*}
        Q = nC_V\Delta T && \text{[volume constant]} \\
        Q = nC_P\Delta T && \text{[pressure constant]}
    \end{align*}
    where $C_V = Mc_V$ and $C_P = Mc_P$ where $M$ is the molecular mass of the gas (g/mol).
\end{definition}
\begin{remark}
    Imagine an ideal gas is slowly heated by $\Delta T$, first at constant volume, then constant pressure. In the constant volume process, no work is done since $\Delta V = 0.$ Thus all the heat $Q_V$ added goes towards increasing internal energy such that $Q_V = \Delta E_{int}.$
    
    However, in the constant pressure process, work \emph{is} done so the added heat $Q_P$ increases the internal energy and does the work $W = P\Delta V_P$ so $\Delta E_{int} = Q_P - P\Delta V_P$. 
    
    Because $\Delta E_{int}$ is equal in both processes ($\Delta T_V = \Delta T_P$), $Q_P - Q_V = P\Delta V \implies nC_P\Delta T - nC_V\Delta T = P(\frac{nR\Delta T}{P})$ from the ideal gas law. This implies $C_P-C_V = R$ which is very accurate to what's obtained experimentally.
\end{remark}
\begin{remark}
    Now, a process done at constant volume on a \emph{monatomic} gas does no work so $nC_V\Delta T = nC_V(T - 0) = Q_V = \Delta E_{int} = N(\frac{1}{2}m\mean{v^2}) = \frac{3}{2}nRT.$ So $C_V = \frac{3}{2}R.$  
\end{remark}
\begin{definition}[Degrees of Freedom]
    Here, we denote \emph{degrees of freedom} to mean the number of independent way molecules can possess energy.
\end{definition}
\begin{example}
    For instance, a diatomic molecule has 5 total degrees of freedom. 3 from translational energy (x,y,z) plus 2 from rotaional kinetic energy (not 3 because the axis along the line of the 2 molecules has such small inertia it is negligible).
\end{example}
\begin{definition}[Principle of Quipartition of Energy]
    Energy is shared equally among the active degrees of freedom and each active degree of a molecule has on average an energy equal to $\frac{1}{2}kT.$
\end{definition}
\begin{note}
    This makes sense as diatomic molecules have energy $\frac{5}{2}nRT$ about $5/3$ times monatomic molecules without degrees of freedom. Hence their $C_V$ being $5/3$ as much. Yet, at extreme low and high temperatures, this diverges such that at low temps, molecules only have translational kinetic energy ($3/2$). And at high temps, molecules also have vibration as 2 degree of freedom (as if from a spring) such that it has kinetic and potential energy ($7/2$). 
\end{note}
\begin{note}
    For solids, we maintain this same spring idea such that molecules have potential and kinetic energy to do with vibration in the $x,y,z$ directions implying 6 degrees of freedom.
\end{note}

\section{Adiabatic Expansion of a Gas}

\begin{remark}
    Take the first law of thermodynamics in an adiabatic process  ($\Delta Q = 0$) for an ideal gas such that $dE_{int} = dQ-dW = -dW = -PdV$. For an ideal gas, $\Delta E_{int} = nC_V\Delta T$ tells us $nC_vdT = -PdV$. Taking the differential of the ideal gas law gives $PdV + VdP = nRdT$ so $nC_V(\frac{PdV+VdP}{nR}) + PdV = 0.$ Rearranging and $C_V+R = C_P$ gives $C_PPdV + C_VVdP = 0$ or $\frac{C_P}{C_V}PdV+VdP=0$. Defining $\gamma = \frac{C_P}{C_V}$ and integrating finally gives $\ln(P)+\gamma\ln(V) = \text{constant}$ which simplifies to $$PV^\gamma = \text{constant}.$$ It's important to note the ideal gas law holds for an adiabatic expansion however $PV$ is clearly not constant implying $T$ must be nonconstant.
\end{remark}

\section{Heat Transfer}

\begin{definition}[Conduction]
    Heat \emph{conduction} can be visualized via molecular collisions such that faster molecules at a heated end collide with slower-moving neighbors and transfer kinetic energy. In metals, collisions of free electrons are mainly responsible for conduction.
\end{definition}
\begin{remark}[Thermal Conductivity]
    Take a uniform cylinder of cross-sectional area $A$ and length $\ell$ such that ends with temperatures $T_1$ and $T_2$ with heat flow $Q$ over a time interval $t$ gives $\frac{Q}{t} = kA\frac{T_1-T_2}{\ell}$ where $k$ is the \emph{thermal conductivity} constant characteristic of the material. The rate of heat flow ($J/s$) is directly proportional to $(T_1-T_2)/\ell$.
\end{remark}
\begin{definition}[Thermal Gradient]
    When $k$ or $A$ cannot be considered constant, we instead take the limit of an infinitesimally thin slab of thickness $dx$ such that our equation becomes $\frac{dQ}{dt} = -kA\frac{dT}{dx}$ where $dT/dx$ is the \emph{temperaute gradient}. Here, the negative sign denotes that heat flow is in direction opposite to the temperature gradient.
\end{definition}
\begin{definition}[Conductors/Insulators]
    Substances for which $k$ is very large conduct heat rapidly and are called good thermal \emph{conductors}. When $k$ is small, the substances are called good thermal \emph{insulators}.
\end{definition}
\begin{definition}[Thermal Resistance]
    The insulating properties of bulding materials are usually specified by $R$-values or \emph{thermal resistance}, defined as $R = \frac{\ell}{k}$ for a thickness $\ell$ of a material. Larger $R$ values imply better insulation.
\end{definition}
\begin{definition}[Convection]
    Though liquids and gases are poor conductors of heat, \emph{Convection} is the rapid process whereby heat flows via the bulk movement of molecules over larger distances.
\end{definition}
\begin{definition}[Forced/Natural Convection]
    For example, \emph{Forced convection} occurs through a fan blowing air while \emph{natural convection} occurs through hot air rising. Hot air or water (in convection currents) rise from buoyancy because heat causes them to expand, decreasing their relative density. 
\end{definition}
\begin{definition}[Radiation]
    Radiation is heat transferred over empty space through EM waves. IR wavelengths are mainly responsible for heating the Earth.
\end{definition}
\begin{definition}[Stefan-Boltzmann equation]
    The rate at which energy leaves a radiating object is $\frac{Q}{t} = \epsilon\sigma AT^4$ where $\sigma$ is the \emph{Stefan-Boltzmann constant} with value \qty{5.67e-8}{W/m^2\cdot K^4}. The factor $\epsilon$, or the \emph{emissivity}, falls between 0 and 1 and is characteristic of the surface of the radiating material.
\end{definition}
\begin{remark}
    A good absorber is \emph{also} a good emitter. Black and dark objects are good emitters with $\epsilon\approx1.$
\end{remark}
\begin{example}
    Take an object of emissivity $\epsilon$ and area $A$ at temperature $T_1$ surrounded by an environment at temperature $T_2$. Because both objects radiate energy to each other so the object absorbs energy proportional to $T^4_2$, the \emph{net} rate of radiant heat flow from the object is given by $\frac{Q}{t} = \epsilon\sigma A(T_1^4-T_2^4).$
\end{example}
\begin{definition}[Solar Constant]
    About \qty{1350}{J} of energy from the Sun strike Earth's atmospher per second per square meter at right angles to the Sun's rays. This number \qty{1350}{W/m^2} is called the solar constant. The atmosphere may absorb as much as 70\% of this energy such that an object of emissivity $\epsilon$ with area $A$ facing the Sun absorbs at a rate about $\frac{Q}{t} = (\qty{1000}{W/m^2})\epsilon A\cos(\theta)$ where $\theta$ is the angle between the Sun's rays and a line perpendicular to the area $A$. Seasons result from this "effective area" $\cos(\theta)$ capture.
\end{definition}
\begin{definition}[Thermography]
    Diagnostic \emph{thermography} uses a thermograph to scan the body and measure the intensity of infrared radiation to detect areas of high metabolic activity.
\end{definition}