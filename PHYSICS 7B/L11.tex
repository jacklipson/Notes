\chapter{Magnetism}

\section{Magnets and Magnetic Fields}

\begin{definition}[Poles]
    \emph{Poles} are where the magnetic effect is strongest. A freely suspended magnet that points towards north is the north pole and one that points south is called the south pole. Like poles repel and unlike attract. \emph{True north} is different from geographic north pole. The angular difference is called \emph{magnetic declination}.
\end{definition}
\begin{note}
    No \emph{magnetic monopole} has ever been observed. Cutting produces new poles.
\end{note}
\begin{definition}[Ferromagnetic]
    Materials which exhibit strong magnetic effects are \emph{ferromagnetic}.
\end{definition}
\begin{remark}
    We can illustrate a magnetic field with field lines so the field direction is tangent to a field at any point and the number of lines per unit area is proportional to the strength of the field.
    
    Magnetic field lines always form closed loops, unlike electric field lines.

    Field lines fringe at edges of two wide poles.
\end{remark}
\begin{note}[Angle of Dip]
    The \emph{angle of dip} is the angle the Earth's magnetic field makes with the horizontal at any point.
\end{note}

\section{Force on an Electric Current in a Magnetic Field}

\begin{definition}[Magnetic Field Force on Current]
    Given magnetic field $\vec{B},$ the force on a wire carrying current $I$ with current direction $\vec{\ell}$ is equal to $$\vec{F} = I\vec{\ell}\times\vec{B}.$$

    If $\vec{B}$ is not uniform, then we can more generally write $d\vec{F}=Id\vec{\ell}\times\vec{B}$ and integrate.

    Note that $\otimes$ represents a field going into a page while $\odot$ represents a field coming out. (X marks the spot).
\end{definition}
\begin{note}[Tesla]
    The SI unit for magnetic field $B$ is the \emph{tesla} T so 1 T = 1 N/A. The \emph{gauus} (G) is also used where \qty{1}{G}=\qty{1e-4}{T}.
\end{note}
\begin{remark}
    If $N$ particles of charge $q$ pass by a given point in time $t$ and travel a distance $\ell$ this constitutes a current $I = Nq/t$ such that $\vec{\ell} = \vec{v}t$ where $\vec{v}$ is the velocity of each particle. Therefore the force on the $N$ particles if $\vec{F} = I\vec{\ell}\times\vec{B} = (Nq/t)(\vec{v}t)\times\vec{B}=Nq\vec{v}\times\vec{B}$ so each particle experiences a force $\vec{F} = q\vec{v}\times\vec{B} = qvB\sin\theta.$
\end{remark}
\begin{remark}[Cyclotron Frequency]
    The time $T$ required for a particle of charge $q$ moving with constant speed $v$ to make one circular revolution is $T = 2\pi r/v = \frac{2\pi m}{qB}$. We call $f = \frac{1}{T} = \frac{qB}{2 \pi m}$ the \emph{cyclotron frequency}.
\end{remark}
\begin{note}
    The \emph{aurora borealis} represents charged ions approaching the Earth from the sun and being increasingly affected by the magnetic field.
\end{note}

\section{Lorentz Equation}

\begin{remark}[Lorentz Equation]
    For a particle of charge $q$ moving with velocity $\vec{v}$ in the presence of both a magnetic field $\vec{B}$ and electric field $\vec{E}$, $$\vec{F} = q(\vec{E} + \vec{v}\times\vec{B}).$$ This is the Lorentz equation.
\end{remark}

\section{Torque on Current Loop \& Magnetic Dipole}

\begin{remark}
    Suppose there exists a rectangular loop of current $I$ with sides $a$ perpendicular to the field and sides $b$ parallel to the field. The loop can spin about the axis lying inbetween and parallel to both sides of length $a$. Thus, the loop will experience net torque $2 * a * I * B * b/2 = IabB = IAB$ where $I$ is the area of the loop. For multiple loops, this becomes $\tau NIAB$. We call the quantity $\vec{\mu} = NI\vec{A}$ the \emph{magnetic dipole moment}. Where $\vec{A}$ points in the direction determined by current and righthand rule. Thus, $$\vec{\tau} = NI\vec{A}\times\vec{B}=\vec{\mu}\times\vec{B}.$$
\end{remark}
\begin{remark}
    Defining $U = 0$ when the loop is perpendicular to the field gives $U = -\mu B\cos\theta = -\vec{\mu}\cdot\vec{B}.$
\end{remark}
\begin{definition}[Galvanometer]
    A \emph{galvanometer} consists of a coil of $N$ loops of wire with an attached pointer suspended in the magnetic field of a permanent magnet. Thus, the torque on the loop is $\tau = NIAB\sin\theta$. This torque is opposed by a spring which has torque $\tau_s = k\phi$ where $\phi$ is the angle through which it is turned and $\theta$ is the angle of the loop relative to the field lines. This implies $\phi = \frac{NIAB\sin\theta}{k}$. To prevent the measured angle $\theta$ from depending on $\theta$ and instead only depending on $I$, a useful meter wraps a galvaonmeter coil around a cylindrical iron core to concentrate the field lines. 
\end{definition}
\begin{definition}[Electric Motor]
    An \emph{electric motor} converts electric energy into mechanical energy. A current-carrying coil of wire in a magnetic field must turn continuously in one direction and is mounted on an iron cylinder called the \emph{rotor} or \emph{armature}. This armature is mounted on a shaft or axle. Because a coil would only rotate past its equilibrium point and then rotate back the opposite way, a current switch is necessary to turn continuously. This can be achieved with a \emph{dc motor} with the use of \emph{commutators and stationary brushes}. On the other hand, an \emph{ac motor} can work withour commutators as its current alternates independently.
\end{definition}
\begin{definition}[Cathode Rays]
    \emph{Cathode rays} are accelerated via a high voltage and pass between a pair of parallel plates built into a cathode ray tube. When another voltage is applied to the plates, an electric field $\vec{E}$ is produced and a pair of coil produce magnetic field $\vec{B}.$ Changing the amount of each field deflects the cathode rays up or down. The force on the rays is $F = evB$. Without an electric field, the rays are bent into a curved path such that $evB = m\frac{v^2}{r}$ so $e/m = v/Br$ which is also equal to $E/B^2r$. These righthandside quantities can all be measured allowing $e/m$ to be determined as \emph{electrons}, or "carriers of electricity." A cathode ray tube applies this high voltage across a small volume of gas in a tube producing a glow at one end.
\end{definition}
\begin{remark}[Oil-Drop Experiment]
    The \emph{oil-drop experiment} of Robert A. Millikan yielded a precise value for the electron's charge. Tiny droplets of mineral oil carrying an electric charge were allowed to fall under gravity between two parallel plates. The electric field $E$ between the plates was adjusted until the drop was suspended in midair implying $qE = mg$ so $q = mg/E$ which could be fully measured, giving us both the charge and mass of an electron.
\end{remark}
\begin{definition}[Quantized Charge]
    Experimental results suggest charges are all integral multiples of $e$ implying electric charge is \emph{quantized}, or exists only in discrete amounts.
\end{definition}
\begin{remark}[The Hall Effect]
    Take a current-carrying wire of diameter $d$ that's held fixed in a magnetic field into the wire from the viewer's perspective. As electrons move right, the magnetic field pushes the electrons downward as $\vec{F_b} = -e\vec{v_d}\times\vec{B}$. This creates a greater density of electrons towards the bottom which creates a potential difference that provides an electric field $\vec{E_H}$ that exerts a force $e\vec{E_H}$ on the moving charges equal and opposite to the magnetic force. This is called the \emph{Hall effect} and the potential, $\mathscr{E}_H = E_Hd = v_dBd$ is called the \emph{Hall emf}. 
\end{remark}
\begin{definition}[Mass Spectrometer]
    A \emph{mass spectrometer} is a device used to measure the masses of atoms. Ions can be produced by heating a sample or through a current. Such ions of mass $m$ and charge $q$ will be shot through a slit $S_1$ and enter a region $S_2$ before another slit where there are perpendicular electric and magnetic fields. To pass through straight, the net force must be 0 so $qE = qvB$ implying $v = E/B$. Thus, only ions of this speed will emerge through slit $S_2$ after which a magnetic field $B'$ is applied forcing the ions to follow a circular path or radius $r$. The field is varied until a detector is triggered or a detector simply measures the diameter of the path of the ion given the known magnetic field. Either way, $qvB' = mv^2/r$ so $m = \frac{qB'r}{v} = \frac{qBB'r}{E}.$
\end{definition}
\begin{remark}[Isotopes]
    Using a mass spectrometer on some materials and ruling out the possibility of impurities implied there must exists different types of atoms with different masses called \emph{isotopes}.
\end{remark}