\chapter{Inductance, Electromagnetic Oscillations, and AC Circuits}

\section{Mutual Inductance}

\begin{definition}[Mutual Inductance]
    Given two coils of wire, a changing current in the first one will induce an emf in the other which is proportional to the rate of change of magnetic flux $\Phi_{21}$ passing through it. If the second coil contains $N_2$ loops, then the \emph{mutual inductance} is $M_{21} = \frac{N_2\Phi_{21}}{I_1}$. and the emf induced in the second coil is $\mathscr{E}_2 = -N_2\frac{d\Phi_{21}}{dt}$ such that $\mathscr{E}_2 = -M_{21}\frac{dI_1}{dt}$.
\end{definition}
\begin{remark}
    In the reverse situation, $\mathscr{E}_1 = -M_{12}\frac{dI_2}{dt}$. It turns out that $M_{12} = M_{21}$ so $M = M_{12} = M_{21}$. Thus, $M$ has unit \emph{henry} such that \qty{1}{H} = \qty{1}{V\cdot s/A} = \qty{1}{\Omega\cdot s}.
\end{remark}
\begin{note}
    Mutual inductance, though useful, can also be a problem. However, it's usually small unless coils with many loops and/or iron cores are involved. Shielded cable (or coaxial) reduces this problem.
\end{note}
\begin{remark}
    Inductance also applies to a single isolated coil of $N$ turns. When a changing current passes through a coil, or solenoid, a changing magnetic flux is produced inside the coil inducing an emf inside the same coil which opposes the change in flux (Lenz's law). 
\end{remark}
\begin{definition}[Self-Inductance]
    The magnetic flux $\Phi_B$ passing through $N$ turns is proportional to the current $I$ in the coil so we define the \emph{self-inductance} $L$ as $L = \frac{N\Phi_B}{I}$. and the corresponding emf induced as $\mathscr{E} = -N\frac{d\Phi_B}{dt} = -L\frac{dI}{dt}$.
\end{definition}
\begin{definition}[Inductor]
    An ac circuit always contains some inductance but often is rather small unless the circuit has many loops or turns. A coil with significant self-inductance is called an \emph{inductor}.
\end{definition}
\begin{definition}[Noninductive Winding]
    Inductance can be minimized in precision resistors by winding insulated wire back on itself in the opposite direction so current produces little magnetic flux. This is \emph{noninductive winding}.
\end{definition}
\begin{remark}
    Given that inductance controls a changing current, for a given $\mathscr{E}$, if the inducance is large, then the change in current will be small implying that if the current is ac, it will also be small. 

    Therefore, inductance is kind of like "resistance" to impede the flow of alternative current, or reactance or impedance.
\end{remark}

\section{Energy Stored in a Magnetic Field}

\begin{remark}
    Given an inductor of inductance $L$ carrying a current $I$, energy is supplied at a rate $P = I\mathscr{E} = LI\frac{dI}{dt}$ such that the work $dW$ done in a time $dt$ is $dW = LIdI$ so the total work to increase the current from 0 to $I$ is, say, $U = W = \frac{1}{2}LI^2$. This is comparable to $U = \frac{1}{2}CV^2$ of a capacitor such that the nergy stored in the electric field between plates is here stored in the magnetic field of an inductor.
\end{remark}
\begin{remark}
    Given a self-inductance $L$ of a long tightly wrapped solenoid containing $N$ turns of wire in its length $\ell$ and area $A$, the internal magnetic field is $B = \mu_0nI = \mu_0 N/\ell I$ so $\Phi_B = BA = \mu_0NIA/\ell$ so $L = \frac{N\Phi_B}{I} = \frac{\mu_0N^2A}{\ell}$.
\end{remark}
\begin{definition}[Energy Density]
    Combining these two remarks, the inductance of an ideal solenoid (end effects ignored) is roughly $U = \frac{1}{2}LI^2 = \frac{1}{2}(\frac{\mu_0N^2A}{\ell})(\frac{B\ell}{\mu_0N})^2 = \frac{1}{2}\frac{B^2}{\mu_0}A\ell$ where this energy is residing in the volume $A\ell$ enclosed by the windings so the \emph{energy density} $u = \frac{1}{2}\frac{B^2}{\mu_0}$.
\end{definition}

\section{LR Circuits}

\begin{remark}
    When a battery of voltage $V_0$ is connected in series with an $LR$ circuit, at the instant the switch is connected and current flows, it is opposed by the induced emf from the inductor by the changing current. There is also a voltage $V = IR$ drop across the resistance in the inductor so the voltage drop across the inductance is reduced and the current increases less rapidly. The current thus rises gradually and approaches a steady value $I_0 = V_0/R$ where there is no more emf in the inductor so all the voltage drop is across the resistance.

    We can show this analytically via the loop rule which says the emfs in the circuit are $V_0 - IR - L\frac{dI}{dt} = 0$ so $\int_{0}^I\frac{dI}{V_0-IR} = \int_0^t\frac{dt}{L}$ so eventually $I = \frac{V_0}{R}(1-e^{-t/\tau})$ where $\tau = \frac{L}{R}$ is the \emph{time constant} of the $LR$ circuit. If the battery is removed, we just get $I = I_0e^{-t/\tau}$ for initial current $I_0$.

    This case is very similar to an $RC$ circuit however, here, the time constant is \emph{inversely} proportional to $R$.
\end{remark}
\begin{definition}[Surge Protector]
    An inductor can act as a \emph{surge protector} for sensitive equipment which may be damaged by high and sudden change in currents by producing an oppposing emf.
\end{definition}

\section{LC Circuits and EM Oscillations}

\begin{remark}
    Take a circuit of just a capacitor with charge $Q_0$ and potential difference $V = Q/C$ connected, via a switch, to an inductor of inductance $L$. At $t=0$ when the switch closes, the capacitor discharges such that $-L\frac{dI}{dt} + \frac{Q}{C} = 0$ so $I = -dQ/dt$ as charge leaves the positive plate to produce a current $I$. Therefore, $\frac{d^2Q}{dt^2} + \frac{Q}{LC} = 0$ so $Q = Q_0\cos(\omega t + \phi)$ for $Q_0$ representing $Q$ at time $t=0$ gives $d^2Q/dt^2 + Q/LC = -\omega^2Q_0\cos(\omega t + \phi) + \frac{Q_0}{LC}\cos(\omega t + \phi) = 0$ so $(-\omega^2+\frac{1}{LC})\cos(\omega t + \phi) = 0$. This is true for all $t$ only if $(-\omega^2 + 1/LC) = 0$ so $$\omega = 2\pi f = \sqrt{\frac{1}{LC}}.$$ Also, we know $I = -\frac{dQ}{dt} = I_0\sin(\omega t + \phi)$ for $I_0 = \omega Q_0=Q_0/\sqrt{LC}$.

    And from the point of energy, at any time $t$, the energy stored in the electric field of the capacitor is $U_E = \frac{1}{2}\frac{Q^2}{C} = \frac{Q_0^2}{2C}\cos^2(\omega t + \phi)$ and similar energy stored in the magnetic field $U_B = \frac{1}{2}LI^2 = \frac{L\omega^2Q_0^2}{2}\sin^2(\omega t + \phi) = \frac{Q_0^2}{2C}\sin^2(\omega t + \phi)$. Therefore, the total energy at any time is $U = U_E + U_B = \frac{1}{2}\frac{Q^2}{C} + \frac{1}{2}LI^2 = \frac{Q_0^2}{2C}$ is constant and conserved.

    This $LC$ circuit is an \emph{LC oscillator} or \emph{electromagnetic oscillation} with the charge $Q$ oscillating back and forth from one plate of the capacitor to another. 
\end{remark}

\section{LRC Circuit}

\begin{remark}
    Take the previous example with the addition that the circuit has some resistance $R$. Given an initial $Q_0$ charge in the capacitor and no battery, if the switch is closed at $t = 0$, then $-L\frac{dI}{dt} - IR + \frac{Q}{C} = 0$. Since $I = -\frac{dQ}{dt}$, this becomes $L\frac{d^2Q}{dt^2} + R\frac{dQ}{dt} + \frac{1}{C}Q = 0$ which is a second-order DE which experiences the same underdamping,critical damping, and overdamping as a harmonic oscillator.
\end{remark}

\section{AC Circuits and Resistance}